\chapter{Related Work}\label{ch:related}
As seen in the previous chapter \ref{ch:problem} the goal of this research is a framework for high bandwidth session based throughput testing.
The term "session based" refers to TCP sessions being build to transfer packets from client to server. 
The term "high bandwidth" refers to links of 40Gb/s and more.

Filling up a link with data is easy, Running IPERF3 in multiple sessions can fill up a link of 100Gb/s with UDP traffic. 
There is some effort made to rewrite IPERF3 to work on top of DPDK. But this did not provide the outcome that was hoped for according to the author of the research paper. \ref{jelte}

\section{transport layer protocols}
The differences in throughput can be found inside the transport layer protocols.
\subsection{UDP}


\subsection{TCP}


\subsection{application specific}


\section{Tools}\label{sec:tools}
A lot of simple tools are available for bandwidth testing. These tools fulfill some purposes that are required within this research.
This research has taken the tools from table \ref{table:tools} to see if there is any use for session bases bandwidth testing on high volumes. 

\begin{table*}[ht]
\centering
\begin{tabular}{|c|c|} \hline
\textbf{Tool} & \textbf{Protocol} \\ \hline
iPerf3 & TCP\& UDP \\ \hline
HPING  & TCP \& UDP\\ \hline
Bonesi & TCP \& UDP\\ \hline
PKTgen Kernel & UDP \\ \hline
PKTGEN DPDK & TCP \& UDP \\ \hline
WARP17 & TCP \& UDP \& HTTP \\ \hline
\end{tabular}
\caption{available tools}
\label{table:tools}
\end{table*} 

\subsection{iPerf3}\label{sub:iperf3}
Iperf3 is an improved version of Iperf that makes it possible the send traffic at higher rates than its predecessors. Since it needs a client and server to generate traffic and the fact that it is hard to generate traffic over 40Gb/s. A small test is performed to see if Iperf3 can be used to generate high bandwidth session based traffic streams. The test and the results can be found in chapter \ref{ch:method}. 

\subsection{PKTgen(kernel module}\label{sub:pktgen}
To generate a single flow of UDP traffic without the need of an application at the other side, PKTGEN from the linux kernel is the way to go. During the test phase preceding this paper, PKTGEN was tested on an Ubuntu machine and on a FreeBSD machine. 
This was done to see if there are any major differences at the kernel level that offer more bandwidth or more packets per second. As it turned out, FreeBSD has the ability to generate 40 million packets per second (pps) from one single thread. Ubuntu needs 6 threads to reach the maximum amount of pps on a 40 Gb link. Other major differences are not found during this research between the different kernels. Therefore the FreeBSD kernel was abandoned after one week. 

\subsection{Bonesi}\label{sub:bonesi}
Bonesi is marketed as a DDOS test tool. It offers the ability to use random sources for TCP syn messages at a high rate. Possibly depleting resources at the targeted device, or Device Under Test (DUT). Resources that can be depleted are memory, since every session needs memory allocation, or session table limits.  

\subsection{HING}\label{sub:hping}


\subsection{DPDK toolkit}\label{sub:dpdk}
The Data Plane Development Kit (DPDK) offers the availability to generate traffic from userland, bypassing the kernel, and directly talking to the hardware. In order to make DPDK run, supported hardware needs to be used. Tools can be created to run on top of DPDK. PKTGEN and WARP17 are two tools that are written on top of DPDK and should provided the ability to generate traffic in high volumes both UDP and TCP based.  

\subsubsection{PKTGEN}\label{subsub:dpdk-pktgen}

\subsubsection{WARP17}\label{subsub:dpdk-WARP17}

\subsubsection{..}\label{}


