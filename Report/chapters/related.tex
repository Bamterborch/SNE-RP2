\chapter{Related Work}\label{ch:related}
As described in chapter \ref{ch:problem}, the goal of this research is the study of a reference model that can be used for high bandwidth session and application based throughput testing.
Tests will be described, which can be used to find the limits of hardware between client and server. 

\section{Transport layer protocols}
The projects goal is to evaluate session-based high-throughput testing, which naturally leads to a focus on the session-oriented transport layer protocol TCP.

%\subsection{UDP}
%When UDP traffic is generated, data is being dumped on the wire without keeping state nor resources need to be claimed for sessions on end hosts, except for the UDP send and receive buffers. 
%Pktgen from the Linux kernel is an 'easy to use' application that is able to generate UDP-only traffic towards a destination. The destination does not need to run an application to receive the data.
%The report written by Turull et al. \cite{turull2016pktgen} about pktgen is an updated report from their work in 2005 and looks at high speed networks. Where high speed for that report is everything over 10Gb/s.  

\subsection{TCP}
When TCP is used to transfer data, session states need to be kept by all the stateful hardware in the path from client to server. 
Congestion control starts to play a role at the senders end to make sure the intermediate devices will not be overloaded. 
The bandwidth filled by the client is dependent on the congestion control mechanism used by the sender.   
Emmerich et al. \cite{emmerich_gallenm¸ller_raumer_wohlfart_carle_2015} published a paper about MoonGen in 2015, MoonGen is capable of generating 120Gb/s and 178.5 Mpps (over multiple 10Gb ethernet interfaces using twelve 2 GHz CPU cores) according to the developers.
Exceeding the criteria for bandwidth and being session-based, it thus meets the requirements for this study.
Therefore, MoonGen looks like a good candidate for OSI layer 2, 3 and 4 testing. 

\subsection{Application specific}
In 2016 research was performed by Malakshmi et al. \cite{mahalakshmi2016study} on different DPDK applications with the purpose of creating a tool for L4 to L7 application testing. 
The result of Malakshmi's research is a tool called T-REX. Their projects goal is to generate stateful traffic up to 10Gb/s. 
However, the main T-REX functionality is Cisco-proprietary and requires a Cisco device to run. The public (free) version is limited in functionality to an extent that it is not interesting for this report.
\section{Tools}\label{sec:tools}
A lot of 'easy to use' tools are available for bandwidth testing. 
These tools each address one or more of the requirements for this study.
Specifically, we consider the tools listed in table \ref{table:tools} in order to assess the suitability of these tools for session based bandwidth testing at high volumes.
iPerf3, hping and BoNeSi are considered 'easy to use' tools.

\begin{table*}[ht]
\centering
\begin{tabular}{|c|c|c|} \hline
\textbf{Tool} & \textbf{Session based} & \textbf{depends on} \\ \hline
iPerf3\cite{iperf} & Yes & Kernel  \\ \hline
hping\cite{hping}  & Yes & Kernel \\ \hline
BoNeSi\cite{bonesi} & Yes & Kernel \\ \hline
%pktgen kernel\cite{pktgen-kernel} & No & Kernel \\ \hline
pktgen DPDK\cite{pktgen-dpdk} & Yes & DPDK \\ \hline
MoonGen\cite{moongen} & Yes & DPDK \\ \hline
WARP\cite{warp} & Yes & DPDK \\ \hline
\end{tabular}
\caption{Packet generation tools}
\label{table:tools}
\end{table*} 

\subsection{iPerf3}\label{sub:iperf3}
IPerf3 is a client-server based tool that allows packet generation.
IPerf3 is an improved version of iPerf that makes it possible the send traffic at higher rates than its predecessors. It needs a client and server to generate traffic and it needs tweaking of kernel parameter to generate traffic over 40Gb/s. Efforts have been made to make it available for DPDK \cite{jelte}. Unfortunately these efforts did not have the success the author was hoping for. 
A small test is performed to see if the kernel based version of iPerf3 can be used to generate high bandwidth session based traffic streams. The test and the results can be found in chapter \ref{ch:experiments}.

\subsection{hping}\label{sub:hping}
Hping was started in 2006. It is a command-line oriented TCP/IP packet assembler. Hping is capable of sending crafted packets to a destination using spoofed IP addresses if necessary. ICMP, UDP, TCP and RAW IP modes are supported. Random source addresses can be used to send requests to simulate a DDOS attack.    

\subsection{BoNeSi}\label{sub:bonesi}
BoNeSi is 'the DDOS botnet simulator' according to its developers. BoNeSi supports ICMP, UDP and TCP (HTTP) flooding attacks from a defined botnet size. Source addresses can be specified in a text file which is used as input. By doing this, one becomes capable of sending crafted packets from different source addresses.  

%\subsection{pktgen(kernel module)}\label{sub:pktgenkernel}
%To generate a single flow of UDP traffic without the need of an application at the other side, pktgen from the Linux kernel is the way to go. During the test phase preceding this paper, pktgen was tested on an Ubuntu machine and on a FreeBSD machine. 
%This was done to see if there where any major differences at the kernel level that offer more bandwidth or more packets per second. 
%As it turned out, FreeBSD has the ability to generate 40 million packets per second (pps) from one single thread. Ubuntu needs 6 threads to reach the maximum pps on a 40 Gb link. 
%Other major differences where not found during this research between the different kernels. Therefore the FreeBSD kernel was abandoned after one week. 
%Pktgen (kernel module) is only capable of generating UDP traffic and therefore, not usable for this research. 

\subsection{Data Plane Development Kit}\label{sub:dpdk}
The Data Plane Development Kit\cite{dpdk} (DPDK) offers the ability to generate traffic from user space, bypassing the kernel and directly talking to the network hardware. In order to make DPDK run, supported NIC's \cite{dpdknic} needs to be used. Applications can be created to run on top of DPDK. Pktgen, MoonGen and WARP are three applications that are written on top of DPDK and should thus to be able to generate traffic in high volumes, both UDP and TCP based. 
%DPDK cannot be run on BSD kernels. For the DPDK experiments all machines where running Ubuntu 16.04 LTS. 

\subsubsection{Pktgen}\label{subsub:dpdk-pktgen}
Pktgen for DPDK is available since May 2013. The developers from DPDK provide Pktgen from the DPDK download page. This makes it a good option for a reference experiment to assess the difference of DPDK compared to kernel-based tests. 

\subsubsection{MoonGen}\label{}
MoonGen was initially released in October 2014. It is designed to generate packets at high speed using a minimum amount of resources from the source. According to the developers it is more efficient than Pktgen. A 10Gb/s link can be filled using only one core. MoonGen builds on libmoon \cite{libmoon} by extending it with features for packet generators such as software rate control and software timestamping. MoonGen offers a report displaying test results \cite{emmerich_gallenm¸ller_raumer_wohlfart_carle_2015}.

\subsubsection{WARP}\label{subsub:dpdk-WARP17}
Juniper WARP was released in May 2016. It allows users to execute performance testing up to layer 7, where however currently only HTTP version 1.1 is supported. IPv6 is not supported at this moment. A server equipped with two Intel® Xeon® E5-2660 v3 processor, 128Gb RAM and two 40 Gb Ethernet interfaces, is supposed to be able to generate around 2 million sessions per second between client and server.  
WARP has not been subject to a research paper. It looks capable of generating high throughput and a high amount of sessions per second up to the application layer.   

