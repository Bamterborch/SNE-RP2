\chapter{Results}\label{ch:results}

From the experiments described in chapter \ref{ch:experiments} a framework was created and tested in chapter \ref{ch:method}. 
The easy to use tools like iPerf3, Hping and Bonesi are usefull for quick tests. But when it comes to representative session basedi application testing these tools do not offer a solution for high speed network. 
As these 'easy to use' tools all require the kernel to talk to the hardware. As described in section \ref{sub:dpdk}, DPDK is able to bypass the kernel and talk to the interface directly. Cores and memory needs to be allocated to generate traffic. By doing this, amounts of 40Gb/s and higher are easy to accomplish.  

\section{DPDK based}
Since none of the use cases requires the kernel based applications, the results of the DPDK based applications are mentioned during the use case perfomance tests on the real world scenario.  

\subsection{PKTGEN}

\todo[inline]{insert images of client, server and network load}

\subsection{WARP}

\todo[inline]{insert images of client, server and network load}


