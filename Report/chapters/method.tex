\chapter{Methodology}\label{ch:method}

In order to create a framework, use cases are formulated.
After this, the correct software can be selected. Only when the use cases are clear and there is an insight in the expected outcome, the correct tools can be selected for the framework. 

\section{Use cases}\label{sec:usecase}

Table \ref{tab:usecases} displays 4 different use cases. All serving a different purpose.
Use case 1 is designed to get the hardware limits of a host. Use case 2 is to get the limits of the  switching and routing hardware.
Use case 3 will get the client and server resources limits. To handle TCP sessions, resources need to be allocated at client and server. 
Use case 4 is designed to get the limits of an application running on top of a kernel. 
When performing these test from a client to a server over a corporate network, using a state full firewall, load balancers, aggregated links and active-standby machines, will show the weakest link in the chain towards the server.  

\begin{table}[]
\centering
\caption{Use cases}
\label{tab:usecases}
\begin{tabular}{|l|l|l|l|}
\hline
NR & Use case                        & DUT            & Explanation                                                                                             \\ \hline
UC1  & \begin{tabular}[c]{@{}l@{}}Bandwidth \\ generation\end{tabular}       & Client         & \begin{tabular}[c]{@{}l@{}}The goal is to see if the client is capable \\ of filling up the link and to reach the \\ maximum amount of pps \end{tabular} \\ \hline
UC2  & Throughput                    & Switch/router  & \begin{tabular}[c]{@{}l@{}}Generate the maximum amount of data in 2 ways \\ to make sure the hardware is able to forward\\ at line rate\end{tabular} \\ \hline
UC3  & TCP based                     & Client/Server  & Get the hardware limitations of the systems  \\ \hline
UC4  & Application             & \begin{tabular}[c]{@{}l@{}}Server and \\intermediate devices\end{tabular}         & \begin{tabular}[c]{@{}l@{}}The clients will overload the server with \\ requests at application level\end{tabular}  \\ \hline
\end{tabular}
\end{table}

\subsection{UC1}
In paragraph \ref{par:pps} from chapter \ref{problem} it is stated that a 40Gb/s card needs to be able to handle 58 million packets with a size of 88 byte to handle a unidirectional stream of 40Gb/s.
Using PKTGEN, a tool on top of DPDK, packets with a minimum size of 88 bytes can be created. This should generate 58 Mpps. To reach the 40Gb/s, large packets have to be create. When the packet size is set to 1500 bytes, there have to 3.3 Mpps to fill up the link. When Jumbo frames are used only 555 thousand packets per second are needed to fill up the link. When using PKTGEN on top of DPDK numbers like these can be reached for TCP traffic. When DPDK is used, the kernel cannot communicate to the devices anymore. So traffic statics cannot be read from the kernel. Monitoring on the switch where the machine is connected to is needed. 

\subsection{UC2}
The backplane of a switch should be able to forward traffic from one port to another at line-speed. This should also be possible when the traffic is routed from one VLAN to the other. Routers should be able to rewrite packets at line rate. Using a client and a server connected to two different ports of the switch and a high load is generated on the ports, preferably the maximum link speed. Monitoring the ports of the switch using SNMP should show the same input and output rate.    

\subsection{UC3}
To get the limitations of the hardware when TCP sessions are used, the kernel has to be bypassed. Using DPDK and WARP17 will show the limits of the hardware under load. As seen in the experimentation chapter \ref{chap:experiments}. The benchmark that was ran on a single machine running as client and server is able to open 1 million session per second, generating 40Gb/s of RAW TCP traffic. 2 tests need to be ran. A RAW TCP test between two different machines, and a test between client and server using HTTP. WARP17 is capable of talking HTTP, a layer 7 protocol. This should be the second test. Monitoring of the amount of successful and failed session has to be done on the clients and server. When the API is used, detailed results can be retrieved.   

\subsection{UC4}
WARP17 is capable of talking HTTP. This use case requires a web server running on one of the machines. The client will use WARP17 to generate traffic towards the web server. The web server has to provide some files of different sizes. Warp can send a request for a certain file with a specified size. Resources of the server will be claimed, By sending a million requests per second the machines will be experiencing a Denial of Service (DOS) like attack.


\section{real world scenario}
The use cases mentioned in section \ref{sec:usecase} need to be tested in a real world scenario. Therefore the infrastructure of a company was used to test this method.
Figure, \ref{fig:nikhefuva} shows the infrastructure of the test between client and server. All tests performed generated traffic for 90 seconds with a gap of 30 seconds before the next tests started from the same category. According to RFC2544 \cite{rfc2544} throughput tests need to take at least 60 seconds. The clients always initiated the sessions to the server.

\begin{figure}
  \includegraphics[scale=0.6]{images/nikhefuva.pdf}
  \caption{Infrastructure of real world scenario.}
  \label{fig:nikhefuva}
\end{figure}

\subsection{UC1}
Using PKTGEN on top of DPDK with a packet size of 64bytes generated a maximum of 42 Mpps towards the server. PKTGEN only sends from one source to one destination, using one sources and destination port. The link from the switch to the router is an aggregated link build from 4x 10Gb/s interfaces. Hashing algorithms decide what interface is used per string. Since the combination is always the same. The packets followed the same path, what ended up in a 10Gb/s stream over one link towards the destination. The packet size was increased after 2 minutes to 400bytes. This is the sweet spot for the amount of packets and bits traveling over the line towards the destination. At 11Mpps and 39Gb/s of traffic onto the line. This ended up in 9Gb/s at the client after hashing over the aggregated lines.  

\todo[inline]{insert images of all the test}

\subsection{UC2}
Due to the hashing problem that was encountered during the first use case. Use case 2 was not performed. Hardware limits cannot be reached by using PKTgen since the limitations of one set of addresses and ports. Therefore, one of the 'easy to use' applications was used. IPerf was stared running in 8 threads. These threads all started to the same port and all coming from an equal port number. The same maximum of 10Gb/s was reached. This new limitation caused by hashing was taken into account. This test failed for this situation

\subsection{UC3}
Sending RAW TCP sessions between 2 servers using WARP17 with different packet sizes should provide a good representation of the capabilities of the device in the path. 
The server ran WARP17 playing a TCP RAW server sending responses of a specified size when requests come in. The client was configured to use 40 thousand ports for the requests targeting 100 ports at the server. This makes a total of 4 Million possible flows between client and server. Request and response sizes are the same at every run. The packet sizes used are, 64, 256, 512, 1024 and 2048 bytes.
During this tests request and reply bitrate should be the same unless one of the devices in between starts failing. As seen in figure \ref{fig:firewallfailure} the amount of traffic coming out of the firewall is not equal to the traffic that goes into the firewall. Failovers to redundant machines happened, BGP sessions died and error where registered in the environment. 
These tests where repeated using the HTTP capabilities of WARP. A good tests to get tests to find the limits of the hardware when HTTP is used as a leyer 7 protocol. Packet size request and response where set to 64, 256, 512, 1024 and 2048 bytes. The same issues where the result of this HTTP test. During the tests the firewalls failed to send there logging information to the management servers. management sessions got cut of. When the tests stopped, everything went back to normal. 

\subsection{UC4}
An NginX server was setup offering a couple of files from a RAMdisk, this is done to make sure disk IO would not become the bottleneck for this test. Using WARP17 at the client side the files where requested from the server. NGINX was capable of delivering 11Gb/s inside test environments.
\todo[inline]{find out what the numbers where during the tests} 

\

%What will be tested?

%Describe the test environment and the upcoming tests
%Why are these test useful? 
%What results are we looking for?
%What are the values/specific settings for these tests?
%What will be monitored on the receiver or sender side.
%Why am I monitoring this.
%What conclusion would I like to draw out of the test results.
 

%TEST environment. 
%2 identical machines running different operating systems at first.
%After preliminary tests the OS might be changed to one OS for final tests.
%In the middle sits a Juniper QFX10k2 as a router/switch.

%Limitations known inside the PCI buss connecting the card. Limited to 40 million packets per second.

%Research from Jelte is useful, it mentions results about about IPERF3 and other specific build tools. 
%I might have to pick up on some of those tools.

%Tools tested.

%PKTgen kernel module. UDP only. 
%Capable of reaching 40Mpps and capable of hitting 40Gb/s

%Ubuntu:
%./pktgen_sample03_burst_single_flow.sh -i eth2 -m 3c:8a:b0:34:2f:f0 -d 10.10.10.10 -s 9000
%-i = interface
%-m = destination mac
%-d = destination address
%-s = size
%-t = threads

%FreeBSD:
%./pkt-gen -i ixl0 -f tx -d 10.10.10.20 -s 10.10.10.10 -S 68:05:ca:32:17:e0 -l 9000
%-i = interface
%-f = function
%-d = destination ip addres
%-s = source ip address
%-S = source mac address
%-l = length
%-R = Rate (amount of packets per second)


%IPERF3.
%capable of hitting 40Gb/s TCP 


