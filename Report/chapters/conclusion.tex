\chapter{Conclusion}\label{ch:conclusion}
From the preliminary tests started in chapter \ref{ch:experiments} and the use cases in chapter \ref{ch:method}, results are acquired and described in chapter \ref{ch:results}.
Conclusion can be drawn from the results. \\
%Flow control and icongestion control used by TCP make sure that links and devices in the path between client and server cannot be overloaded. To get the limitations of hardware, the focus on getting the limitations should be pointing towards packets per second and sessions per second. 
When it comes to generating session based high bandwidth throughput testing, the 'easy to use' tools are not powerful enough. IPerf makes it to 40Gb/s, but does so over only one TCP session per thread. 
Kernel based pktgen was able to generate 40Gb/s and 40Mpps. But it only capable of sending UDP traffic. 
The 'easy to use' tools are useful for quick tests.
DPDK's capability offer much more potential when it comes to session based and application based throughput testing. \\ 

A design conclusion can be drawn from the results as well. A machine that needs to handle a large amount of sessions and sending session based data should not be placed behind a firewall.
The statefull firewalls are not build to handle the amount of sessions per second needed in some cases.  

%The use cases from this report offer a method to tests the infrastructure of a path towards an application.
%When there is a weak link in the path, it will present itself.


\section{Potential}
The Potential of DPDK is still a work in progress. New tools that use the power of DPDK are introduces every year.
Pktgen (2013), MoonGen(2014), T-rex (2015), WARP17(2016).
The possibilities to test up to layer 7 of the OSI model are offered to system and network administrators. 
Current tooling is capable of generating a million session per second using simple server hardware. 
What can be done when powerful servers attached to the Internet with 100Gb/s are used for 'performance testing'?  

\section{Future Work}
During this project an attempt is made to use an IBM Power8 machine to generate traffic at 100Gb/s. Due to problems and time restrictions this attempt was abandoned.
Testing on powerful server hardware in order to reach a 100Gb/s is the next step. 

During this project, HTTP was used for application testing. Support for more protocols need to be added to applications like WARP. 

Currently WARP supports IPv4 only. When IPv6 is supported, the performance should be tested using IPv6. 

DPDK supports multiple NIC's. During the project a small effort is made to start generating traffic over 100Gb/s Mellanox cards.
This was successful up to 60Gb/s TCP traffic, until the system crashed. Real tests need to be run using the tooling discussed in this paper. 
Support and limitations for different 100Gb/s cards need to be researched.   
