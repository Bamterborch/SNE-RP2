\chapter{Problem statement}\label{ch:problem}
Currently there is no way of testing the full chain of devices towards and including an application to there limits without spending a lot of money on commercially available products. 
Since servers can be equipped with 40 and 100Gb/s NIC's, the OS and the application behind the NIC should be able to process the amount of traffic that comes in from the NIC.
The infrastructure from the client to the server needs to be able to handle the traffic to and from the server.
  
A wide variety of tools is available to generate traffic. Some of them offer an easy way of generating packets to put onto the interface in order to fill up the bandwidth. 
Others are able to setup TCP sessions to transfer the traffic, but what tool to use for specific application testing?   
Nikhef wants the ability to test the hardware they might buy up to OSI layer 7 if applicable for the device. 
Until now they where not be able to go beyond layer 4. 
%The purpose of this research is to find out what is needed to generated high volume session based traffic generation.

\section{Session based}\label{sec:sessionbased}
When sending traffic over UDP, data gets dumped onto the link and is transferred to the destination. If the destination is expecting data it will be processed.
When the destination does not know what to do with the data is will be discarded. 
TCP guarantees the delivery of the data as long as the session is alive between the end nodes. 
This is done by acknowledging received data and resending unacknowledged data. These acknowledgments need to be processed by the operating system. 
When all the CPU's of a machine are busy generating packets the overhead of ACK messages will impact the amount of traffic. 
Other techniques like Flow control, Congestion control, and fast retransmission of packets make sure that data is delivered and offered to the higher layer protocol in order. 
These techniques all require resources reservation at the client and the server, but also at statefull devices in the path between client and server. 
When a TCP session is created, memory is reserved by the client and server. In high speeds networks, the amount of reserved memory should be relatively high. 
Ubuntu 16.04 sets the resources reservation by default as follows.

\newpage

\begin{verbatim}
# maximum amount of read memory space in bytes
net.core.rmem_max = 8388608
# maximum amount of write memory space in bytes
net.core.wmem_max = 8388608
# amount of TCP read memory space in bytes (min, default, max)
net.ipv4.tcp_rmem = 131072 1048576 8388608
# amount of TCP write memory space in bytes (min, default, max)
net.ipv4.tcp_wmem = 131072 1048576 8388608
\end{verbatim}  

The default TCP memory allocation is chosen for the first sessions. But when more sessions are opened up the amount of allocated memory drops to the minimum amount. 
Unfinished sessions keep this memory allocated. To deplete the servers resources, the amount of new sessions should be higher than the amount of sessions that get closed due to a time-out.

When it comes to Layer 7 protocols like HTTP, more resources need to be reserved. HTTP sessions need to be saved. Get requests need to be processed, responses need to be generated and send over the session to the user. Normally a web server will cache files that are requested for a specific time using up valuable memory. 
Hosting a dynamic web page requires the web server to generate the page on request which makes the CPU dependencies bigger than just hosting a static web site. 

\section{Specifications}\label{sec:specifications}
Some technical specifications are important and represent leading values for the measurements during this project. 

\paragraph{Throughput}\label{par:throughput}\mbox{}\\
Throughput is the most significant value for generating load. How to fill up the total capacity of the link. When using TCP the header causes 12 bytes more overhead in comparison to a UDP header (a UDP header is 8 bytes).

\paragraph{Packets per second}\label{par:pps}\mbox{}\\
When a link has a speed of 40Gb/s and packets have a size of 88bytes. This includes the inter frame gap, the preamble the data link frame and a VLAN tag. A maximum of 56.8 Million packets per second (Mpps) can be transferred over the link in one direction. When using a 100Gb/s line the theoretical maximum is 142 Mpps.   

\paragraph{Sessions}\label{par:sessions}\mbox{}\\
A session is a unique set of source IP, destination IP, source port and destination port. An established session may involve more than one message in each direction.
The amount of session per second is deterministic for the availability of services behind firewalls. A Firewall needs to keep states of the sessions from source to destination. 
When new sessions to a DTN are opened, the firewall has to process them according to the rule base. When a session is approved, most vendors move it to fast-path processing. 
This is a table with accepted sessions, new traffic in the same session is handled in hardware. This means that only the first packet of a new session is handled in the slow-path.

\paragraph{Application specific traffic}\mbox{}\\
Sandvine provides communication solutions to service providers all over the world. In bi-annual reports they show the traffic baseline. Around 70\% of the traffic is streaming audio and video. At the second place is web browsing and third are marketplace solutions. A lot of on-line services are offered over a web page. 
Therefore it is interesting to look at HTTP traffic as a Layer 7 protocol.

\paragraph{Packetsize}\label{par:packetsize}
According to Murray et all. \cite{murray2012state}  in 2012, 99\% of the traffic inside a corporate network has an MTU size of maximum 1500. When using Jumbo packets (MTU of 9000) the amount of overhead is less because more data can go into on packet. Jumbo packets can help when large amounts of data need to be transfered.  
Therefore jumbo packets are not used during the project unless mentioned otherwise during a test.

\section{Research question}\label{sec:researchquestion}
The problem statement and the specifications lead to the following research question.

\begin{center}
\textit{What is needed to perform high bandwidth session based throughput tests and how to go beyond pure network infrastructure testing?} \\
\end{center}
The term "high bandwidth" references to at least 40Gb/s. \\
The term "session based" references to TCP traffic. \\

