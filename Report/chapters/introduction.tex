\chapter{Introduction}\label{ch:intro}
Internet Service Providers (ISPs) offer high bandwidth links to customers to connect to the Internet. Dutch Universities and institutes are able to connect with redundant 100Gb/s links to there ISP.
Institutes like Nikhef (a tier 1 location for CERN) need to transfer large datasets from CERN. CERN produces around 30 petabytes annually\cite{cerndata}, this makes high uplink speeds critical for research purposes.
For the transfer of large datasets, preferably TCP should be used instead of UDP. TCP offers a reliable connection oriented transfer of the data. 
Mechanisms like flow control and congestion control try to make sure links will not be overloaded to prevent packet loss. The clients need to acknowledge received datagrams to the Data Transfer Node (DTN) in order to keep packets flowing. 
The negotiated window size between client and server determines the amount of data acceptable in flight.  
These techniques make it nearly impossible to bring down a service using high bandwidth only. 


\section{Scope}\label{sec:scope}
% Distributed Denial of Service (DDoS) attacks happen on a daily base. The websites \url{http://www.digitalattackmap.com} and \url{http://map.norsecorp.com} offer an insight in current and passed attacks.

Generating high amounts of throughput is easy to accomplish. 'Easy to use' tools with simple Command Line Interface (CLI) options can create a data stream to a server. For link testing, these tools have sufficient power to fill up lines up to 100Gb/s using UDP traffic. 
The 'easy to use' tools like iPerf\cite{iperf}, Kernel module pktgen\cite{pktgen-kernel}, HPING\cite{hping} and BoNeSi\cite{bonesi} have there own use cases and there own limitations, for simple layer 2 and 3 testing these tools do what they have to.
The increasing demand of bandwidth requires a more sufficient approach for packet generation and application based testing. 

The Data Plane Development Kit\cite{dpdk} (DPDK) introduced by Intel offers users the ability to generate traffic without using the kernel.  
Linux userland applications are able to bypass the kernel and communicate to the hardware directly. Memory, processors and interfaces have to be reserved when DPDK applications are used.
Applications are being build on top of DPDK, utilizing DPDK's functionality to bypass the kernel. MoonGen\cite{moongen}, pktgen\cite{pktgen-dpdk}, and WARP17\cite{warp} are designed with different ideas. Most of the tools offer the ability to test beyond Layer 2 and 3 of the Open Systems Interconnection (OSI) model.
Some are going op to application layer protocols.\\ 

The needs for transferring data over TCP and the increasing demand of bandwidth can result into problems in network infrastructures and applications. 
Statefull devices between client and DTN can become a bottleneck for data transfers.\\ 
To keep a network or a path predictable the limitation of the hardware in the path need to be known.
How can one get to know the limitations of the hardware in the path?
This will be discussed in this paper.  

\section{Document layout}\label{sec:layout}

The layout of this document is as follows: The problem is described in chapter \ref{ch:problem} which is followed by the research question that started this report. To explain what research has been performed in this area chapter \ref{ch:related} shows the researched fields separated per protocol. An experimentation setup is described in chapter \ref{ch:experiments}, this setup is used to see what a selected set of tools had to offer and where they hit their limits.
Chapter \ref{ch:method} will explain some of the use cases and the tools that need to be used to accomplish the goals of the use cases. The method is tested on an environment that was planned to go into production in a short time span. All tests where performed during a scheduled event with approval of the owner of the environment. The results of the real world test can be found in chapter \ref{ch:results}. 
From the results, conclusions are drawn in chapter \ref{ch:conclusion} 

\section{Terminology and Acronyms}\label{sec:terminology}
RFC1242 \cite{rfc1242} explains most of the terminology used in this paper. For your convenience the most useful ones are summed up in table \ref{table:terms}
A list of acronyms used in this paper can be found in appendix \ref{appendix:acronym}.

\begin{table}[]
\centering
\caption{Useful terminology}
\label{table:terms}
\begin{tabular}{|c|l|}
\hline
\textbf{Term}                  & \textbf{Explenation}                                                                                                                                                                                 \\ \hline
Constant Load         & Fixed length frames at a fixed interval time.                                                                                                                                                                    \\ \hline
Data link frame size  & \begin{tabular}[c]{@{}l@{}}The number of octets in the frame from the first octet \\ following the preamble to the end of the FCS, if present, or \\ to the last octet of the data if there is no FCS.\end{tabular} \\ \hline
Inter Frame Gap       & \begin{tabular}[c]{@{}l@{}}The delay from the end of a data link frame, \\ to the start of the preamble of the next data link frame.\end{tabular}                                                                \\ \hline
MTU-mismatch behavior & \begin{tabular}[c]{@{}l@{}}The network MTU (Maximum Transmission Unit) of the output \\ network is smaller than the MTU of the input network, this \\ results in fragmentation.\end{tabular}                        \\ \hline
Overloaded behavior   & When demand exceeds available system resources.                                                                                                                                                                  \\ \hline
Throughput            & \begin{tabular}[c]{@{}l@{}}The maximum rate at which none of the offered frames,are \\ dropped by the device. \end{tabular}                                                                                                                                  \\ \hline
\end{tabular}
\end{table}


