\chapter{Introduction}\label{ch:intro}
Internet Service Providers (ISPs) offer high bandwidth links to customers to connect to the Internet. Dutch Universities and institutes are able to connect with redundant 100Gb/s links to their ISP.
Institutes like Nikhef (a tier 1 location for the large Hadron Collider (LHC) Computing Grid) need to transfer large datasets from CERN. CERN produces around 30 petabytes annually\cite{cerndata}, which makes high-speed links critical for Nikhef's research purposes.
The Nikhef network is designed around a high-speed core that contains only stateless switching and routing devices, and distributed storage systems are directly connected thereto (akin to Data Transfer Nodes (DTN) in a Science Demilitarized Zone).
But besides transfer speed, data integrity is important, and therefore preferably TCP should be used instead of UDP. TCP offers a reliable connection oriented transfer of the data. 
Mechanisms like flow and congestion control try to ensure links will not be overloaded in order to best utilize the available bandwidth. The clients need to acknowledge received datagrams to the server in order to keep packets flowing. 
The negotiated window size between client and server determines the amount of data that can be in flight.  
All these techniques make it nearly impossible to bring down a service using high bandwidth only, even when - like in the case of Nikhef's data transfers - a large number of parallel streams per data file is employed to improve throughput over long-latency links.

\section{Scope}\label{sec:scope}

Generating high volumes of bandwidth is easy to accomplish. 'Easy to use' tools with simple Command Line Interface (CLI) options and using standard operating system (kernel) mechanisms can create a data stream to a server. For link testing, these tools have sufficient power to saturate links well beyond 100Gb/s using UDP traffic. 
The 'easy to use' tools like iPerf\cite{iperf}, Kernel module pktgen\cite{pktgen-kernel}, hping\cite{hping} and BoNeSi\cite{bonesi} have their own use cases and limitations, for simple layer 2 and 3 testing these tools are adequate.
The increasing demand for bandwidth requires a more efficient approach for traffic generation and application based testing. 

The Data Plane Development Kit\cite{dpdk} (DPDK) introduced by Intel offers users the ability to generate traffic without using the kernel TCP/IP and network stack.  
Through DPDK, Linux userland applications are able to bypass the kernel and communicate with the network hardware directly. Memory, processors and interfaces have to be dedicated to DPDK  when DPDK applications are used.
Applications are then build on top of DPDK, utilizing DPDK's functionality to bypass the kernel (and the copying of memory regions inherent in such use). MoonGen\cite{moongen}, pktgen\cite{pktgen-dpdk}, and WARP\cite{warp} each designed based on different ideas offering the ability to test beyond Layer 2 and 3 of the Open Systems Interconnection (OSI) model.
Some are going up to application layer protocols.\\ 

The needs for utilizing TCP to transfer data reliably and the increasing demand for bandwidth can result in problems for both network infrastructures and applications. 
Stateful devices between client and DTN can become a bottleneck for data transfers.\\ 
To have a network or a path behave predictably the limitations of the hardware in the path need to be known.
How can one get to know the limitations of the hardware in the path?
This will be addressed in this paper.  

\section{Document layout}\label{sec:layout}

The structure of this document is as follows: The problem is described in chapter \ref{ch:problem} which is followed by the research question that started this research project. To explain what research has been performed in this area chapter \ref{ch:related} shows the researched fields split by protocol. The experimental setup is described in chapter \ref{ch:experiments}, this setup was used to evaluate what a selected set of tools had to offer as well as their limitations.
Chapter \ref{ch:method} explains the representative use cases and the tools that where used to accomplish the goals of these use cases. The method was put to test on an environment that was planned to go into production - provided the tests described here demonstrated stable and scalable operation. All tests where performed during a scheduled maintenance windows with approval of the owner of the environment. The results of the real world test can be found in chapter \ref{ch:results}. 
From the results, conclusions are drawn in chapter \ref{ch:conclusion}. Appendix \ref{appendix:software} shows the test scripts used for the real world tests. 

\section{Terminology and Acronyms}\label{sec:terminology}
The terminology used in this paper is based on RFC1242 \cite{rfc1242}, with the most relevant terms listed in table \ref{table:terms} for convenience.
A list of acronyms used in this paper can be found in appendix \ref{appendix:acronym}.

\begin{table}[]
\centering
\caption{Useful terminology}
\label{table:terms}
\begin{tabular}{|c|l|}
\hline
\textbf{Term}                  & \textbf{Explanation}                                                                                                                                                                                 \\ \hline
Constant Load         & Fixed length frames at a fixed interval time.                                                                                                                                                                    \\ \hline
Data link frame size  & \begin{tabular}[c]{@{}l@{}}The number of octets in the frame from the first octet \\ following the preamble to the end of the Frame check \\ Sequence (FCS), if present, or to the last octet of the data if \\ there is no FCS.\end{tabular} \\ \hline
Inter Frame Gap       & \begin{tabular}[c]{@{}l@{}}The delay from the end of a data link frame, \\ to the start of the preamble of the next data link frame.\end{tabular}                                                                \\ \hline
MTU-mismatch behavior & \begin{tabular}[c]{@{}l@{}}The network MTU (Maximum Transmission Unit) of the output \\ network is smaller than the MTU of the input network, this \\ results in fragmentation.\end{tabular}                        \\ \hline
Overload behavior   & When demand exceeds available system resources.                                                                                                                                                                  \\ \hline
Throughput            & \begin{tabular}[c]{@{}l@{}}The maximum rate at which none of the offered frames,are \\ dropped by the device. \end{tabular}                                                                                                                                  \\ \hline
\end{tabular}
\end{table}


