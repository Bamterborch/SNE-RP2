\chapter{Introduction}\label{ch:intro}
Internet Service Providers (ISPs) offer high bandwidth links to customers to connect to the Internet. Dutch Universities and institutes are able to connected with redundant 100Gb/s links to there ISP.

Institutes like Nikhef (a tier 1 location for CERN) need to transfer large datasets and have to make calculations on the data. High uplink speeds are critical for research purposes.

A hardware supplier like Juniper is offering hardware with high specifications for data center connectivity. Like a bandwidth of 40Gb/s per port and a maximum of 2 Billion packets per second. An example from Juniper for a QFX10k is given in the quoted part. 

\say{The QFX10000 line of Ethernet switches provides cloud builders and data center operator switch scalable solutions for both core and spine data center deployments. QFX10002-36Q provides 2.88 terabytes of throughput and up to 2 billion pps of forwarding capacity. In native mode, QFX10002-36Q offers 36 ports of 40 gigabit QSFP+.} 

Before buying hardware with these specifications a business has requirements for the system according to the role the hardware is going to fulfill.

\section{Scope}\label{sec:scope}
Distributed Denial of Service (DDoS) attacks happen on a daily base. The websites \url{http://www.digitalattackmap.com} and \url{http://map.norsecorp.com} offer nice maps of current and passed attacks. 


Generating high amounts of throughput is easy to accomplish. Simple tools with a view options can create a data stream to a server. For link testing, these tools have sufficient power to fill up lines up to 100Gb/s. 
The older trusted application like Iperf, Kernel module PKTGEN, HPING and Bonesi have there own use cases and there own limitations, for simple layer 2 and 3 testing these tools are fine.
With increasing bandwidth more and more power is needed for the simple tools to work. Kernels need to be tweaked to reach maximum results from the tooling. 

The Data Plan Development Kit (DPDK) introduced by Intel offers users the ability to generate traffic without using the kernel.  
Linux "Userland" applications are able to bypass the kernel and communicate to the hardware directly. Memory, processors and interfaces are reserved while the tools are used.
Applications are being build on top of DPDK, utilizing DPDK's functionality to bypass the kernel to reach different goals. Moongen, PKTGEN, and WARP17 are designed with different goals. Most of the tools offer the ability to test beyond Layer 2 and 3 of the Open Systems Interconnection (OSI) model. The purpose of this research is to look at the needs for throughput testing at higher layers of the OSI model. \\ 
\todo[inline]{insert references to different tools} 


\section{Document layout}\label{sec:layout}

The layout of this document is as follows: The problem is described in chapter \ref{ch:problem} which is followed by the research question that started this report. To explain what research has been performed in this area chapter \ref{ch:related} shows the researched fields separated per protocol. A tests setup is described in chapter \ref{ch:experiments}, this setup is used to see what all the tools had to offer and where they hit their limits.
Chapter \ref{ch:method} will explain some of the use cases and the tools that need to be used to accomplish the goals of the use cases. The method is testes on an environment that was planned to go into production soon after the tests. All tests where performed during a scheduled event with approval of the owner of the environment. The results of the test can be found in chapter \ref{ch:conclusion}. 
 

\section{Terminology and Acronyms}\label{sec:terminology}
RFC1242 \cite{rfc1242} explains most of the terminology used in this paper. For your convenience the most useful ones are summed up in table \ref{table:terms}

\begin{table}[]
\centering
\caption{Useful terminology}
\label{table:terms}
\begin{tabular}{|c|l|}
\hline
\textbf{Term}                  & \textbf{Explenation}                                                                                                                                                                                 \\ \hline
Constant Load         & Fixed length frames at a fixed interval time.                                                                                                                                                                    \\ \hline
Data link frame size  & \begin{tabular}[c]{@{}l@{}}The number of octets in the frame from the first octet \\ following the preamble to the end of the FCS, if present, or \\ to the last octet of the data if there is no FCS.\end{tabular} \\ \hline
Inter Frame Gap       & \begin{tabular}[c]{@{}l@{}}The delay from the end of a data link frame, \\ to the start of the preamble of the next data link frame.\end{tabular}                                                                \\ \hline
MTU-mismatch behavior & \begin{tabular}[c]{@{}l@{}}The network MTU (Maximum Transmission Unit) of the output \\ network is smaller than the MTU of the input network, this \\ results in fragmentation.\end{tabular}                        \\ \hline
Overloaded behavior   & When demand exceeds available system resources.                                                                                                                                                                  \\ \hline
Throughput            & \begin{tabular}[c]{@{}l@{}}The maximum rate at which none of the offered frames,are \\ dropped by the device. \end{tabular}                                                                                                                                  \\ \hline
\end{tabular}
\end{table}

\begin{table}[]
\centering
\caption{Used Acronyms}
\label{table:acronyms}
\begin{tabular}{l|l}
\hline
\textbf{Acronym}                  & \textbf{Definition}  \\ \hline
pps & packets per second \\ \hline 
Gb/s & Gigabit per second \\ \hline
DPDK & Data Plane Development Kit \\ \hline
ISP & Internet service provider \\ \hline
QSFP & Quad Small Form-factor Pluggable \\ \hline
DOS & Denial Of Service \\ \hline
DDOS & Distributed Denial Of Dervice \\ \hline
FCS & Frame check sequence \\ \hline
MTU & Maximum Transmission Unit \\ \hline
OSI & Open System Interconnection \\ \hline
NIC & Network Interface Card \\ \hline
TCP & Transport Control Protocol \\ \hline
UDP & User Datagram Protocol \\ \hline
ICMP & Internet Control Message Protocol \\ \hline
OS & Operating System \\ \hline
LAN & Local Area Network \\ \hline
VLAN & Virtual LAN \\ \hline
IP & Internet Protocol \\ \hline
HTTP & Hyper Text Transport Protocol \\ \hline
MTU & Maximum Transmission Unit \\ \hline
SNMP & Simple Network management Protocol \\ \hline
RFC & Request For Comment \\ \hline
DUT & Device Under Tests \\ \hline
CPU & Central Processing Unit \\ \hline
UC & Use Case \\ \hline
\end{tabular}
\end{table}

