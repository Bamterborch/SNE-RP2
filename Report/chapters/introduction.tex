\chapter{Introduction}\label{ch:intro}
Internet Service Providers (ISPs) offer high bandwidth links to customers to connect to the Internet. Dutch Universities and research facilities are able to connected with redundant 100Gb/s links to there ISP.

Institutes like Nikhef (a tier 1 location for CERN) need to transfer large datasets and have to make calculations on the data. High uplink speeds are critical for research purposes.

A hardware supplier like Juniper is offering hardware with high specifications for data center connectivity. Like a bandwidth of 40Gb/s per port and a maximum of 2 Billion packets per second. An example from Juniper for a QFX10k is given in the quoted part. 

\say{The QFX10000 line of Ethernet switches provides cloud builders and data center operator switch scalable solutions for both core and spine data center deployments. QFX10002-36Q provides 2.88 terabytes of throughput and up to 2 billion pps of forwarding capacity. In native mode, QFX10002-36Q offers 36 ports of 40 gigabit QSFP+.} 

Before buying hardware with these specifications a business has requirements for the system according to the role the hardware is going to fulfill.

\section{Scope}\label{sec:scope}
Distributed Denial of Service (DDoS) attacks happen on a daily base. The websites \url{http://www.digitalattackmap.com} and \url{http://map.norsecorp.com} offer nice maps of current and passed attacks. 

Generating high amounts of throughput is easy to accomplish, the Data Plan Development Kit (DPDK) introduced by Intel offers users the ability to generate traffic without using kernel driver. 
Linux "Userland" applications are able to bypass the kernel and communicate to the network hardware directly.
Applications are being build, utilizing DPDK's functionality to bypass the kernel to reach different goals. Moongen, PKTGEN, and WARP17 are designed with different goals. \\
\todo[inline]{insert references to different tools} 
Even tough these new applications are the way to go, the older trusted application like Iperf, Kernel module PKTGEN, HPING and Bonesi have there own usecases and there own limitations, they are still useful in lot of the tests performed by administrators. 
This reports describes interesting aspects of the mentioned older and newly created tools.

\section{Document layout}\label{sec:layout}

The layout of this document is straight forward, The problem is described in chapter \ref{ch:problem} which is followed by the research question at hand for this report . To explain what research has been performed in this area chapter \ref{ch:related} goes deeper into some of the useful parts of different applications.
Chapter \ref{ch:method} describes the usecases and the performed tests that are performed to get the results in this report. These results can be found in chapter \ref{ch:conclusion} 
 

\section{Terminology}\label{sec:terminology}
Some terms need to be explained here.......
\todo[inline]{fill in the terms used in the report}
