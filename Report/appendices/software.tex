\thispagestyle{empty}

\chapter{}\label{appendix:software}
\section{Software}

\textbf{Software used during research:}
\begin{itemize}
\item{Ubuntu 16.04 LTS}
\item{FreeBSD 11.0}
\item{DPDK 16.11}
\item{pktgen(dpdk) 3.3.4}
\item{WARP 1.4}
\item{Moongen}
\item{iPerf 3.1.3}
\item{hping 3}
\item{Bonesi V0.3}
\end{itemize}

\begin{landscape}
\section{benchmark test}\label{app:benchmark}

\textbf{Benchmark script for WARP:}
\begin{verbatim}
import sys
import time

from functools import partial

sys.path.append('../../python')
sys.path.append('../../api/generated/py')
sys.path.append('../../ut/lib')

from warp17_api import *

from b2b_setup import *

from warp17_common_pb2    import *
from warp17_l3_pb2        import *
from warp17_server_pb2    import *
from warp17_client_pb2    import *
from warp17_app_raw_pb2   import *
from warp17_app_http_pb2  import *
from warp17_test_case_pb2 import *
from warp17_service_pb2   import *

# 4M sessions
sip_cnt = 1
dip_cnt = 1
sport_cnt = 40000
dport_cnt = 100

sess_cnt = sip_cnt * dip_cnt * sport_cnt * dport_cnt
serv_cnt = dip_cnt * dport_cnt

expected_rate = 1000000
run_cnt = 3

def die(msg):
    sys.stderr.write(msg + ' Should cleanup but we just exit..\n')
    sys.exit(1)

def get_server_test_case(protocol, app, req_size, resp_size):
    l4_scfg = L4Server(l4s_proto=protocol,
                       l4s_tcp_udp=TcpUdpServer(tus_ports=b2b_ports(dport_cnt)))
    app_scfg = {
        RAW: AppServer(as_app_proto=RAW,
                       as_raw=RawServer(rs_req_plen=req_size,
                                        rs_resp_plen=resp_size)),
        HTTP: AppServer(as_app_proto=HTTP,
                        as_http=HttpServer(hs_resp_code=OK_200,
                                           hs_resp_size=resp_size))
    }.get(app)

    return TestCase(tc_type=SERVER, tc_eth_port=1, tc_id=0,
                    tc_server=Server(srv_ips=b2b_sips(1, sip_cnt),
                                     srv_l4=l4_scfg,
                                     srv_app=app_scfg),
                    tc_criteria=TestCriteria(tc_crit_type=SRV_UP,
                                             tc_srv_up=serv_cnt),
                    tc_async=False)

def get_client_test_case(protocol, app, req_size, resp_size):
    l4_ccfg = L4Client(l4c_proto=protocol,
                       l4c_tcp_udp=TcpUdpClient(tuc_sports=b2b_ports(sport_cnt),
                                                tuc_dports=b2b_ports(dport_cnt)))
    rate_ccfg = RateClient(rc_open_rate=Rate(),
                           rc_close_rate=Rate(),
                           rc_send_rate=Rate())

    delay_ccfg = DelayClient(dc_init_delay=Delay(d_value=0),
                             dc_uptime=Delay(),
                             dc_downtime=Delay())

    app_ccfg = {
        RAW: AppClient(ac_app_proto=RAW,
                       ac_raw=RawClient(rc_req_plen=req_size,
                                        rc_resp_plen=resp_size)),
        HTTP: AppClient(ac_app_proto=HTTP,
                        ac_http=HttpClient(hc_req_method=GET,
                                           hc_req_object_name='/index.html',
                                           hc_req_host_name='www.foobar.net',
                                           hc_req_size=req_size))
    }.get(app)

    return TestCase(tc_type=CLIENT, tc_eth_port=0,
                    tc_id=0,
                    tc_client=Client(cl_src_ips=b2b_sips(0, sip_cnt),
                                     cl_dst_ips=b2b_dips(0, dip_cnt),
                                     cl_l4=l4_ccfg,
                                     cl_rates=rate_ccfg,
                                     cl_delays=delay_ccfg,
                                     cl_app=app_ccfg),
                    tc_criteria=TestCriteria(tc_crit_type=CL_ESTAB,
                                             tc_cl_estab=sess_cnt),
                    tc_async=False)

def configure_server_port(warp17_call, protocol, app, req_size, resp_size):
    pcfg = b2b_port_add(eth_port=1, def_gw=Ip(ip_version=IPV4, ip_v4=0))
    b2b_port_add_intfs(pcfg,
                       [
                        (Ip(ip_version=IPV4, ip_v4=b2b_ipv4(eth_port=1, intf_idx=i)),
                         Ip(ip_version=IPV4, ip_v4=b2b_mask(eth_port=1, intf_idx=i)),
                         b2b_count(eth_port=1, intf_idx=i)) for i in range(0, dip_cnt)
                       ])

    if warp17_call('ConfigurePort', pcfg).e_code != 0:
        die('Error configuring port 1!')

    scfg = get_server_test_case(protocol, app, req_size, resp_size)
    if warp17_call('ConfigureTestCase', scfg).e_code != 0:
        die('Error configuring server test case')

def configure_client_port(warp17_call, protocol, app, req_size, resp_size):
    pcfg = b2b_port_add(eth_port=0, def_gw=Ip(ip_version=IPV4, ip_v4=0))
    b2b_port_add_intfs(pcfg,
                       [
                        (Ip(ip_version=IPV4, ip_v4=b2b_ipv4(eth_port=0, intf_idx=i)),
                         Ip(ip_version=IPV4, ip_v4=b2b_mask(eth_port=0, intf_idx=i)),
                         b2b_count(eth_port=0, intf_idx=i)) for i in range(0, sip_cnt)
                       ])

    if warp17_call('ConfigurePort', pcfg).e_code != 0:
        die('Error configuring port 0!')

    ccfg = get_client_test_case(protocol, app, req_size, resp_size)
    if warp17_call('ConfigureTestCase', ccfg).e_code != 0:
        die('Error configuring client test case')


def run_test(protocol, app, req_size, resp_size):
    env = Warp17Env(path='./test_2_perf_benchmark.ini')
    warp17_pid = warp17_start(env=env, exec_file='../../build/warp17',
                              output_args=Warp17OutputArgs(out_file='/tmp/test_2_perf.out'))
    warp17_wait(env=env, logger=LogHelper(name='benchmark',
                                          filename='/tmp/test_2_perf.log'))

    warp17_call = partial(warp17_method_call, env.get_host_name(),
                          env.get_rpc_port(), Warp17_Stub)

    configure_server_port(warp17_call, protocol, app, req_size, resp_size)
    configure_client_port(warp17_call, protocol, app, req_size, resp_size)
    timeout_s = int(sess_cnt / float(expected_rate)) + 2

    if warp17_call('PortStart', PortArg(pa_eth_port=1)).e_code != 0:
        die('Error starting server test cases!')

    if warp17_call('PortStart', PortArg(pa_eth_port=0)).e_code != 0:
        die('Error starting client test cases!')

    time.sleep(timeout_s)

    result = warp17_call('GetTestStatus', TestCaseArg(tca_eth_port=0,
                                                      tca_test_case_id=0))
    if result.tsr_state != PASSED:
        die('Test case didn\'t pass: ' + str(result))

    start_time = result.tsr_stats.tcs_start_time
    end_time = result.tsr_stats.tcs_end_time

    # start and stop ts are in usecs
    duration = (end_time - start_time) / float(1000000)
    rate = sess_cnt / duration
    txr = result.tsr_link_stats.ls_tx_pkts / duration
    rxr = result.tsr_link_stats.ls_rx_pkts / duration
    link_speed_bytes = float(result.tsr_link_stats.ls_link_speed) * 1024 * 1024 / 8
    tx_usage = min(float(result.tsr_link_stats.ls_tx_bytes) * 100 / duration / link_speed_bytes, 100.0)
    rx_usage = min(float(result.tsr_link_stats.ls_rx_bytes) * 100 / duration / link_speed_bytes, 100.0)

    warp17_stop(env, warp17_pid, force=True)
    return (rate, txr, rxr, tx_usage, rx_usage)

def run_test_averaged(descr, protocol, app, req_size, resp_size, run_cnt):
    results = [run_test(protocol, app, req_size, resp_size)
               for i in range(0, run_cnt)]
    avgs = [sum(result, 0.0) / run_cnt for result in zip(*results)]

    # Print as csv
    print '%(descr)s,%(req_size)u,%(resp_size)u,%(rate).0f,%(txr).0f,%(rxr).0f,%(txu).2f,%(rxu).2f' % \
           {
            'descr': descr, 'req_size': req_size, 'resp_size': resp_size,
            'rate': avgs[0], 'txr': avgs[1], 'rxr': avgs[2],
            'txu': avgs[3], 'rxu': avgs[4]
           }

def run():

    # Print csv header
    print 'Description, req_size, resp_size, rate, tx pps, rx pps, tx usage, rx usage'

    # TCP RAW
    tcp_raw_cfg = [(8, 8), (16, 16), (32, 32), (64, 64), (128, 128),
                   (256, 256), (256, 512), (256, 1024), (256, 2048), (256, 4096),
                   (256, 8192), (512, 8192), (1024, 8192), (2048, 8192)]

    for (req_size, resp_size) in tcp_raw_cfg:
        run_test_averaged('TCP request={req!s}b response={resp!s}b'.format(req=req_size,
                                                                           resp=resp_size),
                          TCP, RAW, req_size, resp_size, run_cnt)

    # HTTP
    http_cfg = [(8, 8), (16, 16), (32, 32), (64, 64), (128, 128),
                (256, 256), (256, 512), (256, 1024), (256, 2048), (256, 4096),
                (256, 8192), (512, 8192), (1024, 8192), (2048, 8192)]

    for (req_size, resp_size) in http_cfg:
        run_test_averaged('HTTP request={req!s}b response={resp!s}b'.format(req=req_size,
                                                                            resp=resp_size),
                          TCP, HTTP, req_size, resp_size, run_cnt)

if __name__ == '__main__':
    run()
\end{verbatim}

\textbf{Benchmark script for Moongen:}
\begin{verbatim}
local mg                = require "moongen"
local memory    = require "memory"
local device    = require "device"
local stats             = require "stats"
local log               = require "log"

function configure(parser)
        parser:description("Generates TCP SYN flood from varying source IPs, supports both IPv4 and IPv6")
        parser:argument("dev", "Devices to transmit from."):args("*"):convert(tonumber)
        parser:option("-r --rate", "Transmit rate in Mbit/s."):default(10000):convert(tonumber)
        parser:option("-i --ip", "Source IP (IPv4 or IPv6)."):default("10.0.0.1")
        parser:option("-d --destination", "Destination IP (IPv4 or IPv6).")
        parser:option("-f --flows", "Number of different IPs to use."):default(100):convert(tonumber)
end

function master(args)
        for i, dev in ipairs(args.dev) do
                local dev = device.config{port = dev}
                dev:wait()
                dev:getTxQueue(0):setRate(args.rate)
                mg.startTask("loadSlave", dev:getTxQueue(0), args.ip, args.flows, args.destination)
        end
        mg.waitForTasks()
end

function loadSlave(queue, minA, numIPs, dest)
        --- parse and check ip addresses
        local minIP, ipv4 = parseIPAddress(minA)
        if minIP then
                log:info("Detected an %s address.", minIP and "IPv4" or "IPv6")
        else
                log:fatal("Invalid minIP: %s", minA)
        end

        -- min TCP packet size for IPv6 is 74 bytes (+ CRC)
        local packetLen = ipv4 and 60 or 74

        -- continue normally
        local mem = memory.createMemPool(function(buf)
                buf:getTcpPacket(ipv4):fill{
                        ethSrc = queue,
                        ethDst = "12:34:56:78:90",
                        ip4Dst = dest,
                        ip6Dst = dest,
                        tcpSyn = 1,
                        tcpSeqNumber = 1,
                        tcpWindow = 10,
                        pktLength = packetLen
                }
        end)

        local bufs = mem:bufArray(128)
        local counter = 0
        local c = 0

        local txStats = stats:newDevTxCounter(queue, "plain")
        while mg.running() do
                -- fill packets and set their size 
                bufs:alloc(packetLen)
                for i, buf in ipairs(bufs) do
                        local pkt = buf:getTcpPacket(ipv4)

                        --increment IP
                        if ipv4 then
                                pkt.ip4.src:set(minIP)
                                pkt.ip4.src:add(counter)
                        else
                                pkt.ip6.src:set(minIP)
                                pkt.ip6.src:add(counter)
                        end
                        counter = incAndWrap(counter, numIPs)

                        -- dump first 3 packets
                        if c < 3 then
                                buf:dump()
                                c = c + 1
                        end
                end
                --offload checksums to NIC
                bufs:offloadTcpChecksums(ipv4)

                queue:send(bufs)
                txStats:update()
        end
        txStats:finalize()
end

\end{verbatim}

\section{scripts used during the tests}\label{app:realworld}

\textbf{Script used during the real world pktgen test:}
It starts of sending 64 byte frames, where after 400 byte frames are send to the destination.

\begin{verbatim}
set ip src 0 1.2.3.4/24
set ip dst 0 4.3.2.1
set mac 0 ff:ff:ff:ff:ff:ff
proto tcp 0
set 0 size 64
sleep 30
start 0
sleep 90
stop 0
set 0 size 400
sleep 30
start 0
sleep 90
stop 0
\end{verbatim}

\textbf{Scripts to perform WARP tests during real world test:\\}
\textbf{Server responding to requests from client using WARP}:

\begin{verbatim}
#Set client IP on interface
add tests l3_intf port 0 ip 1.2.3.4 mask 255.255.255.0
add tests l3_gw port 0 gw 1.2.3.1 

add tests server tcp port 0 test-case-id 0 src 1.2.3.4 1.2.3.4 sport 80 180
set tests server http port 0 test-case-id 0 200-OK resp-size 64

add tests server tcp port 0 test-case-id 1 src 1.2.3.4 1.2.3.4 sport 80 180
set tests server http port 0 test-case-id 1 200-OK resp-size 256

add tests server tcp port 0 test-case-id 2 src 1.2.3.4 1.2.3.4 sport 80 180
set tests server http port 0 test-case-id 2 200-OK resp-size 512

add tests server tcp port 0 test-case-id 3 src 1.2.3.4 1.2.3.4 sport 80 180
set tests server http port 0 test-case-id 3 200-OK resp-size 1024

add tests server tcp port 0 test-case-id 4 src 1.2.3.4 1.2.3.4 sport 80 180
set tests server http port 0 test-case-id 4 200-OK resp-size 2048

start tests port 0
show tests ui
\end{verbatim}

\textbf{Client requesting a file from a server (can be run to an NGINX server or to a WARP server)}:
\begin{verbatim}
add tests l3_intf port 0 ip 4.3.2.1 mask 255.255.255.0
add tests l3_gw port 0 gw 4.3.2.254

add tests client tcp port 0 test-case-id 0 src 4.3.2.1 4.3.2.1 sport \ 
10000 50000 dest 1.2.3.4 1.2.3.4 dport 80 180
set tests client http port 0 test-case-id 0 GET "1.2.3.4" \
/files/500K.img req-size 64

set tests timeouts port 0 test-case-id 0 init 30
set tests timeouts port 0 test-case-id 0 uptime 1
set tests timeouts port 0 test-case-id 0 downtime 0
set tests criteria port 0 test-case-id 0 run-time 90

add tests client tcp port 0 test-case-id 0 src 4.3.2.1 4.3.2.1 sport \ 
10000 50000 dest 1.2.3.4 1.2.3.4 dport 80 180
set tests client http port 0 test-case-id 0 GET "1.2.3.4" \
/files/500K.img req-size 256

set tests timeouts port 0 test-case-id 1 init 30
set tests timeouts port 0 test-case-id 1 uptime 1
set tests timeouts port 0 test-case-id 1 downtime 0
set tests criteria port 0 test-case-id 1 run-time 90

add tests client tcp port 0 test-case-id 0 src 4.3.2.1 4.3.2.1 sport \ 
10000 50000 dest 1.2.3.4 1.2.3.4 dport 80 180
set tests client http port 0 test-case-id 0 GET "1.2.3.4" \
/files/500K.img req-size 512

set tests timeouts port 0 test-case-id 2 init 30
set tests timeouts port 0 test-case-id 2 uptime 1
set tests timeouts port 0 test-case-id 2 downtime 0
set tests criteria port 0 test-case-id 2 run-time 90

add tests client tcp port 0 test-case-id 0 src 4.3.2.1 4.3.2.1 sport \ 
10000 50000 dest 1.2.3.4 1.2.3.4 dport 80 180
set tests client http port 0 test-case-id 0 GET "1.2.3.4" \
/files/500K.img req-size 1024

set tests timeouts port 0 test-case-id 3 init 30
set tests timeouts port 0 test-case-id 3 uptime 1
set tests timeouts port 0 test-case-id 3 downtime 0
set tests criteria port 0 test-case-id 3 run-time 90

add tests client tcp port 0 test-case-id 0 src 4.3.2.1 4.3.2.1 sport \ 
10000 50000 dest 1.2.3.4 1.2.3.4 dport 80 180
set tests client http port 0 test-case-id 0 GET "1.2.3.4" \
/files/500K.img req-size 2048

set tests timeouts port 0 test-case-id 4 init 30
set tests timeouts port 0 test-case-id 4 uptime 1
set tests timeouts port 0 test-case-id 4 downtime 0
set tests criteria port 0 test-case-id 4 run-time 90

start tests port 0
show tests ui
\end{verbatim}

\end{landscape}
