\documentclass[11pt,a4paper,notitlepage,openany]{report}
\usepackage{floatrow}
\usepackage[pdftex]{graphicx}
\usepackage[hyphens]{url}
\usepackage[super,comma]{natbib}
\usepackage{listings}
\usepackage{minted}
\usepackage{multicol}
\usepackage{datetime}
\newdateformat{mydate}{\THEDAY\ \monthname[\THEMONTH] \THEYEAR}
\usepackage{csquotes}
\usepackage[linktoc=all]{hyperref}
\usepackage{changepage}
\usepackage{xcolor, colortbl}
\usepackage{tabu}
\usepackage{pdfpages}
\usepackage{amsmath}
\usepackage{pgfplots}
\usepackage[justification=centering]{caption}
\usepackage[titletoc,title]{appendix}
\usepackage{lscape}
\usepackage{geometry}
\usepackage{titling}
\usepackage{algorithm}
\usepackage{algpseudocode}
\usepackage{dirtytalk}
\usepackage{todonotes}
\usepackage{pifont}
\usepackage{titlesec}
\newfloatcommand{capbtabbox}{table}[][\FBwidth]
\usepackage{array}

% Useful custom commands which make text look better
%
% Issue by \<NAME> i.e. \hdfs
%
\usepackage{relsize}
\usepackage{xspace}
\newcommand{\post}{\textsmaller{POST}\xspace}
\newcommand{\tcpdump}{\textsmaller{TCPDUMP}\xspace}
\newcommand{\sslsplit}{\textsmaller{SSL}split\xspace}
\newcommand*\rot{\rotatebox{90}}
\newcommand*\OK{\ding{51}}
\algtext*{EndIf}% Remove "end if" text

\definecolor{butter1}{rgb}{0.988,0.914,0.310}
\definecolor{chocolate1}{rgb}{0.914,0.725,0.431}
\definecolor{chameleon1}{rgb}{0.541,0.886,0.204}
\definecolor{skyblue1}{rgb}{0.447,0.624,0.812}
\definecolor{plum1}{rgb}{0.678,0.498,0.659}
\definecolor{scarletred1}{rgb}{0.937,0.161,0.161}
\pgfplotsset{compat=newest}

\titleformat{\chapter}[display]
{\normalfont\huge\bfseries}{\chaptertitlename\ \thechapter}{20pt}{\Huge}
\titlespacing*{\chapter}{0pt}{5pt}{20pt}

\begin{document}
\begin{titlingpage}
\noindent
\begin{center}
\textsc{\Large Universiteit van Amsterdam}\\[0.2cm]
\textsc{\large System and Network Engineering}\\[0.5cm]

{ \large \bfseries Research Project 2}\\[0.2cm]
{ \LARGE \bfseries Session based high bandwidth throughput testing}\\[0.5cm]
\begin{footnotesize}
Bram ter Borch\\
\texttt{bram.terborch@os3.nl}\\[0.5cm]
{\large \mydate\today}
\end{footnotesize}

\centering
\vspace{2.0cm}
\includegraphics[scale=0.07]{images/uva_logo.png}
\vspace*{2.5cm}
\begin{abstract}
Pushing packets over a link does not seem hard to accomplish. This is true for UDP traffic since it doesn't do session synchronization nor congestion control and does not offer retransmission of data. 
But for a session oriented reliable protocol such as TCP - which ensures that the data sent is actually received by the target like a Data Transfer Node (DTN), it is much more complex to achieve. For this session-based traffic the techniques implemented in a standards-compliant TCP stack make it harder to overload a device in terms of bandwidth.
Thus, both the amount of packets per second and the number of concurrent sessions are the possible limitations this research is focusing on.
Based on a set of use cases, this research investigates the suitability of various load testing tools - with and without the use of the Data Plane Development Kit - and apply these tools to various end-to-end network topologies and hardware systems between a client and a target DTN. 
The report demonstrates that the weakest links in the path can be identified and its load tolerance limits can be reached through these use cases.
\end{abstract}
\end{center}
\end{titlingpage}

\renewcommand{\contentsname}{Table of Contents}
\setcounter{tocdepth}{1}
\tableofcontents
\addtocontents{toc}{\protect\enlargethispage{45mm}}
\begin{flushleft}
\end{flushleft}
\chapter{Introduction}\label{ch:intro}

This paper addresses and describes the path towards obtaining knowledge on the limitations of hardware in the path towards a service, whilst accounting the full path from client to application (Open Systems Interconnect (OSI) layer 7). 
The research considers and explains the importance of the end-to-end user experience

Before an IT department offers a new systems or an application to the users, system and network administrators need to know the limitations of the hardware in the path towards an application in order to set thresholds for monitoring alerts. 
Routers and switches are capable of forwarding packets at line rate, making sure the server running an applications receives all the data destined for the application without unnecessary delays. 
To provide more bandwidth to an application, link aggregation is used to bundle multiple physical links to one logical link.

Generating high bandwidth to saturate a link using a session based protocol is possible. 
Kernel based open source tooling capable of generating data to saturate links well beyond 100Gb/s using UDP traffic, are available.  
Tools like iPerf\cite{iperf}, hping\cite{hping} and BoNeSi\cite{bonesi}i are kernel based and can be used to perform throughput tests up to OSI layer 3.  
%The needs for utilizing TCP to transfer data reliably and the increasing demand for bandwidth can result in capacity problems for both network infrastructures and applications. 

A path towards an application could contain stateful devices like a firewall or a load balancer. 
Stateful devices keep track of sessions, which evidently costs resources. 
These stateful devices in a path towards an application can become a bottleneck when their resources run out and can cause unavailability of services.
Alerting before resources run out is needed to keep a network and the services available.  
Therefore, end-to-end performance tests are needed from a client to a service in order to find the limitations in the infrastructure providing the service.

Institutes like Nikhef (a tier 1 location for the large Hadron Collider (LHC) Computing Grid) need to transfer large datasets from CERN. 
CERN produces around 30 petabytes annually\cite{cerndata}, which makes high-capacity links critical for Nikhef's research purposes.
The Nikhef network is designed around a high-capacity core that contains only stateless switching and routing devices. 
Distributed storage systems are directly connected thereto (akin to Data Transfer Nodes (DTN) in a ScienceDMZ\cite{sciencedmz}).
Besides capacity, data integrity is important, and therefore preferably TCP should be used for data streams between Nikhef and CERN. 
New equipment and services implemented by the engineers of Nikhef need to be tested if they can handle the expected load up to OSI layer 7 (the application layer).

The use of high bandwidth links and a need for session based application layer testing requires a different approach for traffic generation over network paths.

The Data Plane Development Kit\cite{dpdk} (DPDK) introduced by Intel offers a different approach for traffic generation, it does so without using the kernel network stack.  
Through DPDK, Linux userland applications are able to bypass the kernel and communicate with the network hardware directly. Memory, processors and interfaces have to be dedicated to DPDK.
Applications are then built on top of DPDK, utilizing DPDK's functionality to bypass the kernel (and the copying of memory regions inherent in such use). 
MoonGen\cite{moongen}, pktgen\cite{pktgen-dpdk}, and WARP\cite{warp} are designed based on different ideas offering the ability to test up to layer 7 of the OSI model.

%\section{Document layout}\label{sec:layout}
%The structure of this document is as follows: The problem is described in chapter \ref{ch:problem} which is followed by the research question from which this research project originated. To explain what research has been performed in this area chapter \ref{ch:related} shows the related work and provides a reason why this research was needed. The experimental setup is described in chapter \ref{ch:experiments}, this setup was used to evaluate what a selected set of tools had to offer as well as their limitations.
%Chapter \ref{ch:method} explains the usability of the tools and describes some tests that could be conducted using the tools. The tests where carried out on an environment that was planned to go into production - provided the tests described here demonstrated stable and scalable operation. All tests where conducted during a scheduled maintenance windows with approval of the owner of the environment. The results of the real world test can be found in chapter \ref{ch:results}. 
%From the results, conclusions are drawn in chapter \ref{ch:conclusion}. Appendix \ref{appendix:software} shows the test scripts used during tests on production infrastructures.

\section{Terminology and abbreviations}\label{sec:terminology}
The terminology used in this paper is based on RFC1242 \cite{rfc1242}, with the most relevant terms listed in table \ref{table:terms} for convenience.
A list of acronyms used in this paper can be found in appendix \ref{appendix:acronym}.

\begin{table}[]
\centering
\caption{Useful terminology}
\label{table:terms}
\begin{tabular}{|c|l|}
\hline
\textbf{Term}                  & \textbf{Explanation}                                                                                                                                                                                 \\ \hline
Constant Load         & Fixed length frames at a fixed interval time.                                                                                                                                                                    \\ \hline
Data link frame size  & \begin{tabular}[c]{@{}l@{}}The number of octets in the frame from the first octet \\ following the preamble to the end of the Frame check \\ Sequence (FCS), if present, or to the last octet of the data if \\ there is no FCS.\end{tabular} \\ \hline
Inter Frame Gap       & \begin{tabular}[c]{@{}l@{}}The delay from the end of a data link frame, \\ to the start of the preamble of the next data link frame.\end{tabular}                                                                \\ \hline
Overload behavior   & When demand exceeds available system resources.                                                                                                                                                                  \\ \hline
Throughput            & \begin{tabular}[c]{@{}l@{}}The maximum rate at which none of the offered frames, are \\ dropped by the device. \end{tabular}                                                                                                                                  \\ \hline 
\end{tabular}
\end{table}



\chapter{Problem statement}\label{ch:problem}

Currently there is no way of testing the full chain of devices towards and including an application to their limits without spending a lot of money on dedicated commercial load testing products.
Since today servers can be equipped with 40 and 100Gb/s Network Interface Cards (NICs), the OS and the application that is running on top of the OS should be able to process the amount of traffic that comes in from the NIC.

Testing the entire infrastructure using dedicated commercial equipment for every new service is costly and unpractical.
Nikhef is constantly upgrading infrastructure devices to provide the increasing demands of bandwidth. Currently they rely on the hardware specification from the vendors. 
Engineers at Nikhef have a need to find the limitations of the hardware before they put it into production without spending a lot of money and without depending on special commercial hardware.
 
A wide variety of tools is available to generate traffic. Nikhef is interested in TCP traffic specifically.  
Some of the tools are able to setup TCP sessions to transfer data. But what tool is best used for specific application testing?   
Nikhef wants the ability to test hardware they might buy up to the OSI application layer (layer 7) if applicable. 
Until now they are not able to go beyond the transport layer (layer 4).

\section{Session based}\label{sec:sessionbased}
When sending data over UDP, data gets generated and is transported to the destination. 
If the destination IP port is open the data will be forwarded to the application listening on that specific port.
When the destination does not listen on that specific IP port the data will be discarded. 
 
TCP guarantees the delivery of the data as long as the session is alive between the end nodes.
The session needs to be created before data can be exchanged, a three way handshake between source and destination is performed in order to synchronize settings. 
Guaranteed delivery is done by acknowledging received data and resending unacknowledged data. 
Other techniques like Flow control, Congestion control, and fast retransmission of packets ensure that data is delivered in time and offered to the higher layer protocol in the correct order. 
These techniques all require resource reservation at the client and the server, but also at stateful devices along the path between client and server. 

The technique's implemented in TCP all require buffers to store data until the data is acknowledged by the receiving end.
The buffer sizes are reserved by the OS per sessions and negotiated during the three way handshake.  
When the amount of new sessions is higher than the amount of sessions that get closed due to a time-out or a finish statement, server resources are depleted.

When it comes to layer 7 protocols like HTTP, more resources need to be reserved. HTTP sessions, HTTP state and application state need to be saved. Get requests need to be processed, a response needs to be generated and sent over the session to the user. Normally a web server will cache files that are requested for a specific time using up valuable memory. Hosting a dynamic web page requires the web server to generate the page on request which makes the CPU utilization higher than just hosting a static web site. 

\section{Specifications}\label{sec:specifications}
For the design of tests and the interpretation of the results, specific technical constraints in the 'real world' operating environment and several key implementation parameters are important for this project.

\paragraph{Overloading}\label{par:overload}\mbox{}\\
All the techniques used by TCP make sure an Ethernet link will not be overloaded with congestion as a result. 
In order to find the weakest link in the path towards a service, data should be send using the expected maximum capacity of the devices in the path.
When client and server are connected with 40Gb/s interfaces. One should send data using the full capacity of the links.
This could result in overloading a device in the path, this immediately shows that the intermediate device is the weakest link.

When data can be send at the links full capacity, there might be other limitations in the hardware used in the path.  
For systems administrators it is important for monitoring purposes that these limitations are known.

\paragraph{Throughput}\label{par:throughput}\mbox{}\\
Throughput is the most significant value for generating load: how to best utilize the total capacity of a link. When using TCP the header causes 12 bytes more overhead in comparison to a UDP header (a UDP header is 8 bytes). But in return TCP offers techniques that will not overload a device. 

\paragraph{Packet size}\label{par:packetsize}\mbox{}\\
Ethernet is used during this research, therefore all references to packet sizes are based on Ethernet standards\cite{ethernet_frame_2017} with IP and TCP headers included.
Due to collision detection the minimum payload inside an Ethernet frame is 46 bytes.
The Ethernet frame and the payload combined have a minimum size of 64 bytes. 
A VLAN tag is excluded which adds 4 bytes.
A packet is always preceded with a 8 byte preamble.
An Inter Frame Gap between the packets is used to separate the packets. This is a 12 byte gap. 
This makes a total of at least 88 bytes of data from the beginning of a packet to the beginning of a new packet. Figure \ref{fig:juniperethernetframe} gives a representation of an Ethernet packet. 

\begin{figure}[H]
  \includegraphics[scale=1]{images/ethernetframe.jpg}
  \caption{Representation of an Ethernet frame used during this report.}
  \label{fig:juniperethernetframe}
\end{figure}

\paragraph{Packets per second}\label{par:pps}\mbox{}\\
When a link has a speed of 40Gb/s and packets have a minimum size of 88bytes (which includes the inter frame gap, the preamble the data link frame and a VLAN tag) a maximum of 56.8 Million packets per second (Mpps) can be transferred over the link in one direction. When using a 100Gb/s line the theoretical maximum is 142 Mpps.   

\paragraph{Sessions}\label{par:sessions}\mbox{}\\
A TCP session is a unique tuple of source IP, destination IP, source port and destination port. An established session may involve more than one message in each direction.
The amount of sessions per second is a determining factor for the availability of services behind firewalls. A firewall needs to keep track of the states of the sessions from source to destination. 
When new sessions to a server are opened, the firewall has to process them according to the rule base. When a session is approved, most vendors move it to fast-path processing. 
This is a table with accepted sessions, allowing traffic in the same session to be handled in hardware. This means that only the first packet of a new session is handled in the slow-path and the limitations of a firewall can be found in the amount of new sessions per second.

\paragraph{Application specific traffic}\mbox{} \\
According to Sandvine\cite{phenomena_2017},(a global communications solutions service provider that published bi-annual traffic baseline reports), around 70\% of the traffic is streaming audio and video. 
Dynamic Adaptive Streaming over HTTP (DASH)\cite{dash} is used to stream video and audio over HTTP. Netflix uses DASH to deliver content to the users. You Tube is accessible over HTTP.
This makes HTTP a suited protocol to use for layer 7 tests during the research.  

\paragraph{Packet size}\label{par:packetsize}\mbox{}\\
According to Murray et all. \cite{murray2012state}  in 2012, 99\% of the traffic inside a corporate network has an MTU size of maximum 1500 bytes. When using Jumbo packets\cite{alliance_2017} (Best practice is an MTU of 9000 bytes towards clients\cite{jet}) the amount of overhead is less because more data can go into on packet. 
More data inside a packet results in less packets. Therefore, less system overhead during the transfer of data. 
Jumbo packets are helpful when large amounts of data need to be transfered between 2 nodes.  
Jumbo packets are not used during the project because according to Murray et all. less than one percent of transferred data has an MTU larger than 1500 bytes. 
Exceptions can be made during a test to see if hardware limits can be reached.

\section{Research question}\label{sec:researchquestion}
The problem statement and the specifications lead to the following research question.

\begin{center}
\textit{What is needed to perform high bandwidth session based throughput tests and how to go beyond pure network infrastructure testing?} \\
\end{center}
The term "high bandwidth" references to at least 40Gb/s. \\
The term "session based" references to TCP traffic. \\


\chapter{Related Work}\label{ch:related}
As described in chapter \ref{ch:problem}, the goal of this research is the study of a reference model that can be used for high bandwidth session and application based throughput testing.
Tests will be described, which can be used to find the limits of hardware between client and server. 

\section{Transport layer protocols}
The projects goal is to evaluate session-based high-throughput testing, which naturally leads to a focus on the session-oriented transport layer protocol TCP.

%\subsection{UDP}
%When UDP traffic is generated, data is being dumped on the wire without keeping state nor resources need to be claimed for sessions on end hosts, except for the UDP send and receive buffers. 
%Pktgen from the Linux kernel is an 'easy to use' application that is able to generate UDP-only traffic towards a destination. The destination does not need to run an application to receive the data.
%The report written by Turull et al. \cite{turull2016pktgen} about pktgen is an updated report from their work in 2005 and looks at high speed networks. Where high speed for that report is everything over 10Gb/s.  

\subsection{TCP}
When TCP is used to transfer data, session states need to be kept by all the stateful hardware in the path from client to server. 
Congestion control starts to play a role at the senders end to make sure the intermediate devices will not be overloaded. 
The bandwidth filled by the client is dependent on the congestion control mechanism used by the sender.   
Emmerich et al. \cite{emmerich_gallenm¸ller_raumer_wohlfart_carle_2015} published a paper about MoonGen in 2015, MoonGen is capable of generating 120Gb/s and 178.5 Mpps (over multiple 10Gb ethernet interfaces using twelve 2 GHz CPU cores) according to the developers.
Exceeding the criteria for bandwidth and being session-based, it thus meets the requirements for this study.
Therefore, MoonGen looks like a good candidate for OSI layer 2, 3 and 4 testing. 

\subsection{Application specific}
In 2016 research was performed by Malakshmi et al. \cite{mahalakshmi2016study} on different DPDK applications with the purpose of creating a tool for L4 to L7 application testing. 
The result of Malakshmi's research is a tool called T-REX. Their projects goal is to generate stateful traffic up to 10Gb/s. 
However, the main T-REX functionality is Cisco-proprietary and requires a Cisco device to run. The public (free) version is limited in functionality to an extent that it is not interesting for this report.
\section{Tools}\label{sec:tools}
A lot of 'easy to use' tools are available for bandwidth testing. 
These tools each address one or more of the requirements for this study.
Specifically, we consider the tools listed in table \ref{table:tools} in order to assess the suitability of these tools for session based bandwidth testing at high volumes.
iPerf3, hping and BoNeSi are considered 'easy to use' tools.

\begin{table*}[ht]
\centering
\begin{tabular}{|c|c|c|} \hline
\textbf{Tool} & \textbf{Session based} & \textbf{depends on} \\ \hline
iPerf3\cite{iperf} & Yes & Kernel  \\ \hline
hping\cite{hping}  & Yes & Kernel \\ \hline
BoNeSi\cite{bonesi} & Yes & Kernel \\ \hline
%pktgen kernel\cite{pktgen-kernel} & No & Kernel \\ \hline
pktgen DPDK\cite{pktgen-dpdk} & Yes & DPDK \\ \hline
MoonGen\cite{moongen} & Yes & DPDK \\ \hline
WARP\cite{warp} & Yes & DPDK \\ \hline
\end{tabular}
\caption{Packet generation tools}
\label{table:tools}
\end{table*} 

\subsection{iPerf3}\label{sub:iperf3}
IPerf3 is a client-server based tool that allows packet generation.
IPerf3 is an improved version of iPerf that makes it possible the send traffic at higher rates than its predecessors. It needs a client and server to generate traffic and it needs tweaking of kernel parameter to generate traffic over 40Gb/s. Efforts have been made to make it available for DPDK \cite{jelte}. Unfortunately these efforts did not have the success the author was hoping for. 
A small test is performed to see if the kernel based version of iPerf3 can be used to generate high bandwidth session based traffic streams. The test and the results can be found in chapter \ref{ch:experiments}.

\subsection{hping}\label{sub:hping}
Hping was started in 2006. It is a command-line oriented TCP/IP packet assembler. Hping is capable of sending crafted packets to a destination using spoofed IP addresses if necessary. ICMP, UDP, TCP and RAW IP modes are supported. Random source addresses can be used to send requests to simulate a DDOS attack.    

\subsection{BoNeSi}\label{sub:bonesi}
BoNeSi is 'the DDOS botnet simulator' according to its developers. BoNeSi supports ICMP, UDP and TCP (HTTP) flooding attacks from a defined botnet size. Source addresses can be specified in a text file which is used as input. By doing this, one becomes capable of sending crafted packets from different source addresses.  

%\subsection{pktgen(kernel module)}\label{sub:pktgenkernel}
%To generate a single flow of UDP traffic without the need of an application at the other side, pktgen from the Linux kernel is the way to go. During the test phase preceding this paper, pktgen was tested on an Ubuntu machine and on a FreeBSD machine. 
%This was done to see if there where any major differences at the kernel level that offer more bandwidth or more packets per second. 
%As it turned out, FreeBSD has the ability to generate 40 million packets per second (pps) from one single thread. Ubuntu needs 6 threads to reach the maximum pps on a 40 Gb link. 
%Other major differences where not found during this research between the different kernels. Therefore the FreeBSD kernel was abandoned after one week. 
%Pktgen (kernel module) is only capable of generating UDP traffic and therefore, not usable for this research. 

\subsection{Data Plane Development Kit}\label{sub:dpdk}
The Data Plane Development Kit\cite{dpdk} (DPDK) offers the ability to generate traffic from user space, bypassing the kernel and directly talking to the network hardware. In order to make DPDK run, supported NIC's \cite{dpdknic} needs to be used. Applications can be created to run on top of DPDK. Pktgen, MoonGen and WARP are three applications that are written on top of DPDK and should thus to be able to generate traffic in high volumes, both UDP and TCP based. 
%DPDK cannot be run on BSD kernels. For the DPDK experiments all machines where running Ubuntu 16.04 LTS. 

\subsubsection{Pktgen}\label{subsub:dpdk-pktgen}
Pktgen for DPDK is available since May 2013. The developers from DPDK provide Pktgen from the DPDK download page. This makes it a good option for a reference experiment to assess the difference of DPDK compared to kernel-based tests. 

\subsubsection{MoonGen}\label{}
MoonGen was initially released in October 2014. It is designed to generate packets at high speed using a minimum amount of resources from the source. According to the developers it is more efficient than Pktgen. A 10Gb/s link can be filled using only one core. MoonGen builds on libmoon \cite{libmoon} by extending it with features for packet generators such as software rate control and software timestamping. MoonGen offers a report displaying test results \cite{emmerich_gallenm¸ller_raumer_wohlfart_carle_2015}.

\subsubsection{WARP}\label{subsub:dpdk-WARP17}
Juniper WARP was released in May 2016. It allows users to execute performance testing up to layer 7, where however currently only HTTP version 1.1 is supported. IPv6 is not supported at this moment. A server equipped with two Intel® Xeon® E5-2660 v3 processor, 128Gb RAM and two 40 Gb Ethernet interfaces, is supposed to be able to generate around 2 million sessions per second between client and server.  
WARP has not been subject to a research paper. It looks capable of generating high throughput and a high amount of sessions per second up to the application layer.   


\chapter{experiments}\label{ch:experiments}
All experiments described in this chapter are executed in a test environment at NIKHEF. The test environment is displayed in image \ref{fig:testenv}.

\todo[inline]{insert image of test environment}

Three identical servers (A, B and C)  are used to perform the tests. Table \ref{tab:testmachines} contains the specifics of the three servers. 
These servers are all connected to a Juniper QFX10k2 (device S). This switch has 32 40Gb/s QFSP ports. 
Some of these ports are configured as a 100Gb/s interface. 
During the experiments 2 extra machines are introduced into the network. Both containing 100Gb/s Mellanox cards.
Table \ref{tab:ppc-intel} contains the information about the extra machines.   

Device A is always the receiving end of the test. Depending on the tests the source can be machine B, C or B and C.
For possible +40Gb/s tests, one of the 100Gb/s machines (D or E) can be used.

Machine M is a Simple Network Management Protocol (SNMP) server. This SNMP server query's the machines every second for status. As seen in the image of the test environment \ref{fig:testenv}. SNMP is active on a different interface of the devices.  

\sebsection{Iperf3}
When Using Iperf3, packets can be generated with a simple one liner.
To make it to 40Gb/s, eight separate threads have to be started when MTU size is set to default.
When an MTU size of 9000 (jumbo packets) is used, 2 threads where needed to generate 40Gb/s. 
So this is useful inside company networks that are under control of the person using iPerf3. 

\todo[inline]{insert images of client, server and network load) 
\todo [inline] {insert image of test}

\subsection{Hping}

\todo[inline]{insert images of client, server and network load)
\subsection{Bonesi}

\todo[inline]{insert images of client, server and network load)
\subsection{DPDK}

\todo[inline]{insert images of client, server and network load)

\chapter{Methodology}\label{ch:method}

In order to create a framework to get the limitations of the hardware and connecting infrastructure up to layer 7, use cases are formulated.
Only when the use cases are clear and there is an insight in the expected outcome, the correct applications can be selected for the framework.
The expected outcome is derived from the experiments. 

\section{Use cases}\label{sec:usecase}

Table \ref{tab:usecases} displays four different use cases. All serving a different purpose.
Use case 1 is designed to get the hardware limits of the NIC inside a host. Use case 2 is to get the limits of the  switching and routing hardware of the DUT.
Use case 3 will get the client and server limits up to layer 4. To handle TCP sessions, resources need to be allocated at client and server. 
Use case 4 is designed to get the limits of an application running on top of a kernel.  
Use cases 1 and 2 will show local limitations. 
When executing use cases 3 and 4 from a client to a server over a corporate network, using a statefull firewall, load balancers, aggregated links and active-standby machines.They will show the weakest link in the infrastructure towards the server.  
It should be noted that monitoring is in important part for showing the weakest link. 

\begin{table}[]
\centering
\caption{Use cases}
\label{tab:usecases}
\begin{tabular}{|l|l|l|l|}
\hline
NR & Use case                        & DUT            & Goal                                                                                             \\ \hline
UC1  & \begin{tabular}[c]{@{}l@{}}Bandwidth \\ generation\end{tabular}       & Client         & \begin{tabular}[c]{@{}l@{}}The goal is to see if the client is capable \\ of filling up the link and to reach the \\ maximum amount of pps \end{tabular} \\ \hline
UC2  & Throughput                    & Switch/router  & \begin{tabular}[c]{@{}l@{}}Generate the maximum amount of data in 2 ways \\ to make sure the hardware is able to forward\\ at line rate\end{tabular} \\ \hline
UC3  & TCP based                     & Client/Server  & Get the hardware limitations of the systems  \\ \hline
UC4  & Application             & \begin{tabular}[c]{@{}l@{}}Server and \\intermediate devices\end{tabular}         & \begin{tabular}[c]{@{}l@{}}The clients will overload the server with \\ requests at application level\end{tabular}  \\ \hline
\end{tabular}
\end{table}

\subsection{UC1}
In paragraph \ref{par:pps} it is stated that a 40Gb/s card needs to be able to handle 58 million packets with a size of 88 byte to handle a unidirectional stream of 40Gb/s.
Using PKTGEN, a tool on top of DPDK, TCP packets with a minimum size of 64 bytes (without inter frame gap, preamble and VLAN tag) can be created. This should generate 58 Mpps. To reach the 40Gb/s, larger packets have to be created. When the packet size is set to 1500 bytes, there have to be 3.3 Mpps to fill up the link. When Jumbo frames are used only 555 thousand packets per second are needed to fill up the link. When using PKTGEN on top of DPDK numbers like these can be reached using TCP traffic. When DPDK is used, the kernel cannot communicate to the devices anymore. So traffic statics cannot be read from the kernel. Monitoring on the switch ports connecting the servers is needed. 

\subsection{UC2}
The backplane of a switch should be able to forward traffic from one port to another at line speed. This should also be possible when the traffic is routed from one VLAN to the other. Routers should be able to rewrite packets at line rate. To test this a client and a server need to be connected to two different ports of the switch in a different segment or VLAN a high load should be generated from the client to the server, preferably the maximum link speed. Monitoring the ports of the switch using SNMP should show the same input rate on port 1 compared to the output rate of port 2.    

\subsection{UC3}
To get the limitations of the servers hardware when TCP is used, the kernel has to be bypassed. Using DPDK and WARP17 will show the limits of the hardware under load. As seen in the experiments in chapter \ref{ch:experiments}. The benchmark that was ran on a single machine running as client and server show the ability to open 1 million session per second, generating a maximum of 20Gb/s of RAW TCP traffic. Two tests need to be executed. A RAW TCP test between two different machines, and a test between client and server using HTTP. WARP17 is capable of talking HTTP, a layer 7 protocol. This should be the second test. Monitoring of the amount of successful and failed session has to be done on the client and server. When the API is used, detailed results can be retrieved as shown during the experimentation phase.   

\subsection{UC4}
WARP17 is capable of talking HTTP. This use case requires a web server running on one of the machines (the DTN). The client will use WARP17 to generate HTTP GET requests towards the DTN. The web server has to provide some files of different sizes. Warp can send a request for a certain file with a specified request size. Resources of the server and statefull devices on the path towards the server will be claimed opening up the sessions, By sending a million requests per second the machines will be experiencing a Denial of Service (DOS) like attack.


\section{Real world scenario}
The use cases mentioned in section \ref{sec:usecase} have to be tested in a real world scenario to see if they are useful. Therefore the infrastructure of a company was used to test this method.
Figure, \ref{fig:nikhefuva} shows the simplified infrastructure for the test between client and server. All tests performed, generated traffic for 90 seconds with a gap of 30 seconds before the next tests started in the same category. According to RFC2544 \cite{rfc2544} throughput tests need to take at least 60 seconds. The clients always initiated the sessions to the DTN.

\begin{figure}
  \includegraphics[scale=0.6]{images/nikhefuva.pdf}
  \caption{Infrastructure of real world scenario.}
  \label{fig:nikhefuva}
\end{figure}

\subsection{UC1}
Using PKTGEN on top of DPDK with a packet size of 64bytes generated a maximum of 42 Mpps towards the server. PKTGEN only sends from one source to one destination, also using just one source and destination port. The link from the switch to the router at the client side is an aggregated link build up from 4x 10Gb/s interfaces. Hashing algorithms decide what interface is used per stream. Since the combination is always the same and a switch's default hashing algorithm uses layer 2 addresses as input. The packets followed the same path, what ended up in a single 10Gb/s link towards the destination. The packet size needs to be increased after 2 minutes to 400bytes. This is the sweet spot for the amount of packets and bits traveling over the line towards the destination with the potential of sending 11Mpps and 39Gb/s of traffic to the server.  

\subsection{UC2}
Due to the hashing settings that cannot be changed without effect on the total environment, use case 2 does not need to be performed. A total of 40Gb/s cannot be reached.

\subsection{UC3}
Sending RAW TCP sessions between 2 servers using WARP17 with different packet sizes should provide a good representation of the capabilities of the servers and the intermediate devices in the path from source to destination. 
The server should be running WARP17, acting as a TCP RAW server listening on 100 ports. Sending responses of a specified size when requests come in. The client could be configured to use 40 thousand ports for the requests, targeting the 100 ports at the servers side. This makes a total of 4 million possible flows between client and server. Request and response sizes can be the same at every run. The request/response sizes depend on the link speed, in this case proper values are, 64, 256, 512, 1024 and 2048 bytes. 

\subsection{UC4}
A web server needs to be setup offering a couple of files from a RAMdisk, this should be done to make sure disk IO will not become the bottleneck for the tests. Using WARP17 at the client side requesting the files from the DTN. NGINX was capable of delivering 11Gb/s inside the experimentation environment. This test will get the limits of the infrastructure towards the DTN or from the DTN itself running the application. 

\

%What will be tested?

%Describe the test environment and the upcoming tests
%Why are these test useful? 
%What results are we looking for?
%What are the values/specific settings for these tests?
%What will be monitored on the receiver or sender side.
%Why am I monitoring this.
%What conclusion would I like to draw out of the test results.
 

%TEST environment. 
%2 identical machines running different operating systems at first.
%After preliminary tests the OS might be changed to one OS for final tests.
%In the middle sits a Juniper QFX10k2 as a router/switch.

%Limitations known inside the PCI buss connecting the card. Limited to 40 million packets per second.




%Ubuntu:
%./pktgen_sample03_burst_single_flow.sh -i eth2 -m 3c:8a:b0:34:2f:f0 -d 10.10.10.10 -s 9000
%-i = interface
%-m = destination mac
%-d = destination address
%-s = size
%-t = threads

%FreeBSD:
%./pkt-gen -i ixl0 -f tx -d 10.10.10.20 -s 10.10.10.10 -S 68:05:ca:32:17:e0 -l 9000
%-i = interface
%-f = function
%-d = destination ip addres
%-s = source ip address
%-S = source mac address
%-l = length
%-R = Rate (amount of packets per second)




\chapter{Results}\label{ch:results}

From the experiments described in chapter \ref{ch:experiments} a framework was created and tested in chapter \ref{ch:method}. 
The easy to use tools like iPerf3, Hping and BoNeSi are useful for quick tests. But when it comes to representative session based application testing these tools do not offer the solution for high speed networks. 
As these 'easy to use' tools all require the kernel to talk to the hardware. As described in section \ref{sub:dpdk}, DPDK is able to bypass the kernel and talk to the interface directly. Cores and memory needs to be allocated to generate traffic.  

\section{Infrastructure}
The network at 'company x' as it is visualized in figure \ref{fig:testenv} is a simplified representation.
The detailed infrastructure used during the real world test is displayed in figure \ref{fig:companyx} 
Since all the network hardware is redundant, the figure gets more complicated.
The server is connected to a data center layer that is spread over two physical locations and connected by DWDM.
To minimize the unnecessary traffic between the data centers on overlay technique is used.  
When traffic does not arrive at the destination, detailed measurements are needed at every device for every link to determine where the traffic gets dropped.  

\begin{figure}[] 
  \includegraphics[scale=0.4]{images/companyx.pdf}
  \caption{Detailed visualization of real world scenario.}
  \label{fig:companyx}
\end{figure}

\section{DPDK based}
Since none of the use cases require the 'easy to use' tools, simply due to the fact these tools did not hit the performance required for this research. 
The results of the DPDK based applications are explained during the use case performance tests on the real world scenario.
  
\subsection{pktgen}
Pktgen was used to get the limitations from the test hardware, it was already known that the maximum amount of pps is a hardware limitation inside the PCI Express Bus. 
The maximum link speed can be reached when packets of 400bytes are used. To get an idea of the throughput over the entire path from client to server. Pktgen is used to setup a UDP stream from the client to the server. 
The source IP address and port towards the destination IP address and port are the same. Since aggregated interfaces are used from the test switch to the router. The hashing algorithm in the switch made sure only one link was filled up by this test. 
Figure \ref{fig:surftest} shows a maximum throughput of 7 GB at 20:40. This was transported to company x and delivered to router1. From here it is important to know what lines are used to transport the traffic towards the firewall.
Figures \ref{fig:testrealusageae112} and \ref{fig:testrealusageae113} display the graphs of the traffic that went out of the interfaces connecting router1 to the firewall cluster and the links connecting the firewalls to the DCI switches inside the data center. The white gaps between send and received data, is traffic that got lost during the execution of the use cases.
   
\begin{figure}
  \includegraphics[scale=0.6]{images/test-link-usage.png}
  \caption{Bandwidth utilization of real world tests.}
  \label{fig:surftest}
\end{figure}

\begin{figure}
  \includegraphics[scale=0.6]{images/real-ae112.png}
  \caption{Bandwidth utilization of links between router1, firewall1 and DC1.}
  \label{fig:testrealusageae112}
\end{figure}

\begin{figure}
  \includegraphics[scale=0.6]{images/real-ae113.png}
  \caption{Bandwidth utilization of links between router1, firewall2 and DC2.}
  \label{fig:testrealusageae113}
\end{figure}



% 20:42 - 20:50 PKTGEN UDP en TCP
% 20:54 - 21:00 WARP - NGINX
% 21:21 - 21:30 WARP - RAW TCP
% 21:42 - 22:00 WARP - HTTP

\subsection{WARP}
From the benchmark results performed on WARP in chapter \ref{ch:experiments} it is known WARP is capable of generating almost a Million sessions per second. WARP was used to retrieve a 500K file form an NGINX web server running on server A.
The file was placed on a RAM disk to make sure disk IO will not be the bottleneck of the performance tests. 
This test is executed on 20:50 and ran until 21:00. Request size increased every 90 seconds with an interval of 30 seconds between the tests.
Request sizes during this test: 64, 256, 512, 1024 and 2048 bytes. 
Figure \ref{fig:realnginx} displays that the amount of traffic leaving the server goes above 4Gb while only 0.6 Gb of requests are coming in. 
This matches the values shown in figure \ref{fig:testrealusageae112}. \\
Rate limiting at the NGINX server made sure nothing broke. The services got depleted at the machine, this made the service unavailable for other users.

\begin{figure}
  \includegraphics[scale=0.35]{images/real-nginx.png}
  \caption{Bandwidth utilization server A during NGINX test, measurements from servers perspective }
  \label{fig:realnginx}
\end{figure}


At 21:20 the next test is started. Generating a maximum amount of RAW TCP sessions from Client to Server both running WARP, Using 32GB memory and all cores to generate traffic.
Request and response sizes chosen for this tests are: 64, 256, 512, 1024 and 2048 bytes. 
Since WARP needs control of the interfaces, the server will not display any traffic on the interfaces. 
The measurements all came from the uplink and downlink interfaces of the Firewall. 
Sessions statistics where not registered due to logging problems in the firewall environment. 
Chapter \ref{ch:experiments} shows the benchmark of 1 million sessions per second are generated, no differences are shown at different packets sizes. 
Only 10\% of the link is utilized at the starting point of the test as displayed in \ref{fig:rawtcplink}. The graph in figure \ref{fig:testrealusageae112} doesn't display any traffic. 
The graph in figure \ref{fig:testrealusageae113} displays the load. When the test started, the active firewall crashed and a failover to the passive machine is the result. Log messages retrieved from connected routers also display BGP session failures from the former active firewall. 
This graph also displays the difference between traffic send from the Router to the firewall and the traffic received by the downstream switch. 
Input is 3Gb and output is in the range of 200 - 300 Mb. Clearly the firewalls are not capable of handling this amount of sessions. 
When the test was finished, the firewalls started to recover functionality. \\ 

The firewall environment was restored to the way it was in the beginning of the first test.
At 21:40 the last test is started between server and client, again both running WARP. Generating the maximum amount of HTTP sessions using 32Gb and all available cores.
Generating a GET request and responding with a 200-OK message. 
Message sizes for the tests are : 64, 256, 512, 1024 and 2048 bytes. Request and response size are equal. 
With the information from the benchmark in chapter \ref{ch:experiments} the limitations of client and server are known. 
From the tests at 21:20 it is known the machine will die when to many sessions per second come in. The same behavior is expected during this test.
Figure \ref{fig:testrealusageae112} displays the graph showing 4Gb going into the firewall and only 500Mb arriving at the other side.



\chapter{Conclusion}\label{ch:conclusion}
From the tool assessment performed in chapter \ref{ch:experiments} and the tests described in chapter \ref{ch:method}, results are gathered and described in chapter \ref{ch:results}.
The results, executed according to current standards and best practices and reproduced as a matter of course, allow to draw conclusions on tool suitability and the necessary characteristics of high-bandwidth session based throughput tests in a real-world environment. \\
When it comes to generating session based high bandwidth throughput testing, DPDK should be used in combination with pktgen in order to reach hardware limitations. 
Using DPDK in combination with WARP creates the possibility to generate traffic at the application layer. 
The kernel based tools could not provide the goals this research was looking for. 

During this research, the limitations of the hardware used in the experimental setup were found by executing the tests. 
The tests can be used as guideline to find the hardware limits in the path from client to server. 
T1 revealed  a limitation for the amount of packets per second in the PCI Express bus. 
T2 was used to get the maximum possible bandwidth from client to server, the hashing settings were shown to be a limitation in the setup as expected.
T3 revealed the hardware limits for client and server with regards to the amount of sessions and bandwidth usage.
Next to the hardware  limits T3 revealed a limitation of a stateful firewall in the path towards the destination, the overload behavior of the firewall is also known by executing this test.\\ 
T4 stressed an application to get the performance limits for this specific application.
 
These tests should be used to get a better insight in performance requirements for high bandwidth infrastructure predictability.
Combining DPDK with pktgen and WARP can reveal the limits of an infrastructure.
To perform high bandwidth session based throughput tests up to layer 3, pktgen on top of DPDK is capable of reaching hardware limits.
For application layer link testing, WARP is the framework to use. Support for other applications need to be added to WARP but the start looks promising.
The use of different kernels did not show any major differences in the results of executed kernel based application tests during this research. 

The exact limit of the firewall was not found, it is only known that one server using WARP can generate the amount of sessions per second to make the system fail.
By increasing the amount of sessions step by step we could have pinpointed the amount of sessions where the firewall started failing.

%When looking at the results of the real world test one could argue if it is useful to equip a web server with a 40Gb/s interface when the maximum HTTP performance using WARP is below 20Gb/s.
%When running NGINX as a web server, a 40Gb/s NIC is not economically viable for the hardware used during this research. 

\section{Suitability of the Data Plane Development Kit}
The Data Plane Development Kit is still a work in progress, and today we see only the beginning of its potential being harnessed for network load testing. New tools that use the power of DPDK are introduced every year:
Pktgen (2013), MoonGen(2014), T-rex (2015), WARP(2016).
The possibilities to test up to layer 7 in the OSI model are now becoming available to system and network administrators. 
Current tooling is capable of generating a million session per second using simple server hardware. 
DPDK applications supporting IPv6 and multiple application layer protocols are needed to improve infrastructures in order to offer services. 

\section{Future Work}
The hashing algorithm used at Nikhef's core network limited the performance tests to 10Gb/s. Using 4 clients or changing the hashing settings should result in more bandwidth utilization. 
By doing this, the other limitation this research was looking for such as the amount of packets per second being a bottleneck can be reached. Further analysis on the Data Center Infrastructure layer at company x can be performed. 
%A close collaboration with the engineers of company x should help them to overcome the problems before the network goes into production. 

During this project an attempt was  made to use an IBM Power8 machine (server E) to generate traffic at 100Gb/s. Because of problems during compilation and memory allocation this attempt had to be abandoned due to time constraints.

This project used HTTP version 1.1 for application testing. Support for more protocols need to be added to WARP to make it more powerful. 
Currently WARP supports IPv4 only. When IPv6 is supported, the performance should be tested using IPv6. 
Monitoring in WARP should be improved, currently the API provides the only way of getting detailed results.
NGINX is made available for DPDK recently. Running WARP towards a DPDK NGINX server should provide the capabilities of NGINX when it does not rely on kernel interrupts. \\

DPDK supports multiple NICs. During the project an effort was made to start generating traffic over 100Gb/s Mellanox cards.
This was successful up to 60Gb/s TCP traffic, until the system crashed for reasons that could not be determined within the scope of this project. 
The proposed tests in this research paper need to be run using the Mellanox cards. 
Support and limitations for different 100Gb/s cards need to be researched.

The Generation 3, 8 lane PCI express cards are a limiting factor as shown in this paper. 
Further investigations could look into the limitations of 16-lane PCIe and its associated scaling behavior.

Intel offers a guide to improve the throughput for the XL710 40Gb/s card for the Linux kernel \cite{intellinuxguidxl710}. 
This guideline provides kernel settings that might improve the results for the kernel based tools.
During this research the guideline was not used to improve the kernel settings, the reason for this is that the settings are dependent on the application that is ran on top of the kernel. 
This research, due to time constraints focused on the DPDK tooling. Tweaking the kernel for all the tools in table \ref{table:tools} is out of scope for this research while it would be very interesting to know if the proposed settings from the guide will impact the performance of the kernel based tools. 


% UNCOMMENT THIS TO PRINT ONLY CITED REFERENCES
\nocite{*}
\bibliographystyle{plainnat}
\bibliography{report}
\begin{appendices}
\appendix
\thispagestyle{empty}

\chapter{}\label{appendix:acronym}

\begin{table}[H]
\centering
\caption{Used Acronyms}
\label{table:acronyms}
\begin{tabular}{l|l}
\hline
\textbf{Acronym}                  & \textbf{Definition}  \\ \hline
DTN & Data Transfer Node \\ \hline
CLI & Command Line Interface \\ \hline
PPS & packets per second \\ \hline
Gb/s & Gigabit per second \\ \hline
DPDK & Data Plane Development Kit \\ \hline
ISP & Internet service provider \\ \hline
QSFP & Quad Small Form-factor Pluggable \\ \hline
DOS & Denial Of Service \\ \hline
DDOS & Distributed Denial Of Service \\ \hline
FCS & Frame check sequence \\ \hline
MTU & Maximum Transmission Unit \\ \hline
OSI & Open System Interconnection \\ \hline
NIC & Network Interface Card \\ \hline
TCP & Transport Control Protocol \\ \hline
UDP & User Datagram Protocol \\ \hline
ICMP & Internet Control Message Protocol \\ \hline
OS & Operating System \\ \hline
LAN & Local Area Network \\ \hline
VLAN & Virtual LAN \\ \hline
IP & Internet Protocol \\ \hline
HTTP & Hyper Text Transport Protocol \\ \hline
MTU & Maximum Transmission Unit \\ \hline
SNMP & Simple Network management Protocol \\ \hline
RFC & Request For Comment \\ \hline
DUT & Device Under Tests \\ \hline
CPU & Central Processing Unit \\ \hline
UC & Use Case \\ \hline
API & Application Programmable Interface \\ \hline
EVPN & Ethernet Virtual Private Network \\ \hline
DWDM & Dense Wave Division Multiplexing \\ \hline
\end{tabular}
\end{table}

\thispagestyle{empty}

\chapter{}\label{appendix:software}

\textbf{Software used during research:}
\begin{itemize}
\item{Ubuntu 16.04 LTS}
\item{DPDK 16.11}
\item{pktgen(dpdk) 3.3.4}
\item{WARP 1.4}
\item{Moongen}
\item{iPerf 3.1.3}
\item{hping 3}
\item{Bonesi V0.3}
\item{pktgen(kernel) version depends on the kernel}
\end{itemize}
\textbf{Scripts to perform WARP tests:\\}
\textbf{Server responding to requests from client using WARP}:
\begin{verbatim}
#Set client IP on interface
add tests l3_intf port 0 ip 145.18.40.36 mask 255.255.255.248
add tests l3_gw port 0 gw 145.18.40.33 

add tests server tcp port 0 test-case-id 0 src 145.18.40.36 145.18.40.36 sport 80 180
set tests server http port 0 test-case-id 0 200-OK resp-size 64

add tests server tcp port 0 test-case-id 1 src 145.18.40.36 145.18.40.36 sport 80 180
set tests server http port 0 test-case-id 1 200-OK resp-size 256

add tests server tcp port 0 test-case-id 2 src 145.18.40.36 145.18.40.36 sport 80 180
set tests server http port 0 test-case-id 2 200-OK resp-size 512

add tests server tcp port 0 test-case-id 3 src 145.18.40.36 145.18.40.36 sport 80 180
set tests server http port 0 test-case-id 3 200-OK resp-size 1024

add tests server tcp port 0 test-case-id 4 src 145.18.40.36 145.18.40.36 sport 80 180
set tests server http port 0 test-case-id 4 200-OK resp-size 2048

start tests port 0
show tests ui
\end{verbatim}
\textbf{Client requesting a file from a server (can be run to an NGINX server or to a WARP server)}:
\begin{verbatim}
add tests l3_intf port 0 ip 192.87.92.0 mask 255.255.255.254
add tests l3_gw port 0 gw 192.87.92.1

add tests client tcp port 0 test-case-id 0 src 192.87.92.0 192.87.92.0 sport \ 
10000 50000 dest 145.18.40.36 145.18.40.36 dport 80 180
set tests client http port 0 test-case-id 0 GET "145.18.40.36" \
/files/500K.img req-size 64

set tests timeouts port 0 test-case-id 0 init 30
set tests timeouts port 0 test-case-id 0 uptime 1
set tests timeouts port 0 test-case-id 0 downtime 0
set tests criteria port 0 test-case-id 0 run-time 90

add tests client tcp port 0 test-case-id 1 src 192.87.92.0 192.87.92.0 sport \ 
10000 50000 dest 145.18.40.36 145.18.40.36 dport 80 180
set tests client http port 0 test-case-id 1 GET "145.18.40.36" \
/files/500K.img req-size 256

set tests timeouts port 0 test-case-id 1 init 30
set tests timeouts port 0 test-case-id 1 uptime 1
set tests timeouts port 0 test-case-id 1 downtime 0
set tests criteria port 0 test-case-id 1 run-time 90

add tests client tcp port 0 test-case-id 2 src 192.87.92.0 192.87.92.0 sport \ 
10000 50000 dest 145.18.40.36 145.18.40.36 dport 80 180
set tests client http port 0 test-case-id 2 GET "145.18.40.36" \
/files/500K.img req-size 512

set tests timeouts port 0 test-case-id 2 init 30
set tests timeouts port 0 test-case-id 2 uptime 1
set tests timeouts port 0 test-case-id 2 downtime 0
set tests criteria port 0 test-case-id 2 run-time 90

add tests client tcp port 0 test-case-id 3 src 192.87.92.0 192.87.92.0 sport \ 
10000 50000 dest 145.18.40.36 145.18.40.36 dport 80 180
set tests client http port 0 test-case-id 3 GET "145.18.40.36" \
/files/500K.img req-size 1024

set tests timeouts port 0 test-case-id 3 init 30
set tests timeouts port 0 test-case-id 3 uptime 1
set tests timeouts port 0 test-case-id 3 downtime 0
set tests criteria port 0 test-case-id 3 run-time 90

add tests client tcp port 0 test-case-id 4 src 192.87.92.0 192.87.92.0 sport \ 
10000 50000 dest 145.18.40.36 145.18.40.36 dport 80 180
set tests client http port 0 test-case-id 4 GET "145.18.40.36" \
/files/500K.img req-size 2048

set tests timeouts port 0 test-case-id 4 init 30
set tests timeouts port 0 test-case-id 4 uptime 1
set tests timeouts port 0 test-case-id 4 downtime 0
set tests criteria port 0 test-case-id 4 run-time 90

start tests port 0
show tests ui
\end{verbatim}
\textbf{Script to perform pktgen test:}
\begin{verbatim}
set ip src 0 192.87.92.2/30
set ip dst 0 145.18.40.36
set mac 0 3c:8a:b0:34:2f:f0
proto tcp 0
set 0 size 64
sleep 30
start 0
sleep 90
stop 0
set 0 size 400
sleep 30
start 0
sleep 90
stop 0
\end{verbatim}


\end{appendices}
\end{document}
