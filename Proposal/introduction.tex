\section{Introduction}
The backbone of the internet is being routed by a protocol known as the Border Gateway Protocol (BGP). Without this protocol, networks of various ISPs and institutions would not be able to communicate to each other in a cost-efficient manner. However, BGP has not been designed with security in mind. This results in worldwide BGP hijacks on a possible daily basis. This phenomenon can be described as advertising IP prefixes or even Autonomous System (AS) numbers that don't belong to the advertiser. New additions to BGP are being designed as Secure BGP (sBGP) and BGPsec. As sBGP uses certificates to validate the origin of route advertisements, AS and advertised prefixes it need to be implemented on all BGP routers worldwide. Older routers need to be replaced when an ISP wants to implement sBGP. The ability to Hijack prefixes is reintroduced when an ISP chooses not to implement sBGP.  BGPSEC is supposed to overcome all issues currently known within BGP and sBGP. However, according to Vervier, there's not yet a software implementation available of this protocol\cite{vervier2015mind}.\\

BGP hijacking has been a hot topic from 2007 on, and the detection of hijacked prefixes and AS numbers has been subject to some research papers \cite{hu2007accurate}\cite{zheng2007light}. However, proposed methods of early stage detection of BGP hijacks were not found sufficient. On July 25th 2015, the Dutch newspaper \textit{"De Volkskrant"} reported a hijack of an IP prefix of the Dutch Ministry of Foreign Affairs. Due to these findings, the minister of Foreign Affairs had to answer questions of the Dutch House of Representatives \cite{koenders2015response}. Therefore, the National Cyber Security Center (NCSC) would like to detect these kind of malicious activities earlier on.
