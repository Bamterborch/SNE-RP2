\section{Methodology}
In order to establish 40Gb/s of session based data a couple of things need to be chosen.
First an operating system should be chosen to perform the tests on. Preliminary to this research some days are used to explore the pitfalls around bandwidth generation. Two identical systems are used to do performance tests.
One running Ubuntu and the other running FreeBSD. 
The kernel addon PKTgen and Netmap for FreeBSD is used to generate UDP traffic to different systems. The amount of 40Gb/s was easily reached on both systems.
But there was a main difference regarding the amount of packets per second (pps). The Ubuntu machine peaked around 4Mpps per thread where the FreeBSD machine was capable of hitting 42 Mpps in one single thread (the limit of the interfaces PCI bus) \cite{chelsio}.
The Ubuntu machine is capable of hitting 42 Mpps when seven separate threads are started. 

Therefore preliminary tests will be performed over multiple OS-es. Since the limitation of two identical machines and running into time restriction, one operating system will be selected to perform final tests that have there base at the preliminary tests.

When the results of the final tests are available, recommendations are written and a framework is produced that can be used as a guidance to build a software suite to perform overall tests in high bandwidth infrastructures.

The goal of this research is to find the current limitations in session based throughput testing and creating a framework that allows engineers at Nikhef to fully test there systems to the maximum specifications.

Terms from RFC1242 \cite{rfc1242} will be used during this research and techniques from RFC2544 \cite{rfc2544} will be considered to be used during this research. 
