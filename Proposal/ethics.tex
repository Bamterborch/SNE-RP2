\section{Ethics}
The hijack detection system will merely use publicly available information. Therefore, it may query resources like WHOIS or BGP metadata databases. In order to gather BGP updates, the monitoring system needs to be attached to a BGP peering network. As we will only perform passive monitoring using public information, no systems of third parties will be disrupted. As this project aims to enable monitoring of hijacked BGP prefixes, anyone could be able to use this system to monitor any BGP prefix. As all of this information is already public, we don't foresee ethical issues. Information regarding hijacked prefixes discovered during the project will be disclosed to the NCSC.
